% (This file is included by thesis.tex; you do not latex it by itself.)

\begin{abstract}

% \begin{center}
%     \textit{``An ancient time\\
%     When compass needles lost their way; \\
%     Finding North from South." \\}
%     \vspace{10pt}
%     S. W. Bogue
% \end{center}

Over 2000 years ago, one of humanity's earliest records of harnessing magnetism unfolded in ancient China, where lodestones (i.e. magnetic rocks) were used in telling directions in geomancy and divination practices. Compass made its way to Europe through the Silk Road, which granted European mariners the ability to tell their heading regardless of weather conditions or visibility constraints at sea. The improved navigation for mariners contributed to the Age of Discovery. Scientific advances through the following centuries have made profound insights into the fundamental principles of electromagnetism. Today, our continued understanding and manipulation of the electromagnetic principles lie at the core of modern technology and life. 

It is because compass needles are made of magnetic materials that tend to align with Earth's magnetic field at surface which is dominantly dipolar and stable such that a compass can help navigate today. In the rock records, there are numerous microscopic compasses that are carrying ancient magnetic records through deep time. These microscopic compasses are magnetic minerals such as iron-titanium oxides and iron sulfides that occur rather ubiquitously in igneous, sedimentary, and metamorphic rocks. Being rooted in understanding of rock forming processes and the physics of electromagnetism, the study of paleomagnetism utilizes these magnetic compasses in rocks to explore Earth history topics.

In this dissertation, I find my applications of paleomagnetism in revealing the characteristics of magmatism in the Midcontinent Rift system in the Lake Superior region and in the southwestern Laurentia 1.1 billion years ago, shedding new lights in the strength of the geodynamo ca. 1092 million years ago, and furthering the development of global paleogeography through the end of the Mesoproterozoic and the earliest Neoproterozoic. 

\subsection{Late Mesoproterozoic magmatism in the Midcontinent Rift and the southwestern Laurentia large igneous province}

The ancient North American craton, Laurentia, first formed in the Paleoproterozoic Era when a series of collisional orogenies culminating the Trans-Hudson orogeny led to the amalgamation of Archean provinces \cite{Hoffman1988a, Whitmeyer2007a}. The craton continued to grow through the rest of the Paleoproterozoic and into the Mesoproterozoic via accretionary orogenesis along its margin. In the latest Mesoproterozoic (ca. 1110 to 1080 Ma) the large intracontinental Midcontinent Rift, which was co-located with a large igneous province (LIP) \cite{Swanson-Hysell2021a} led to extension within the Archean Superior province and adjacent Paleoproterozoic provinces to the south \cite{Cannon1992a}. Protracted magmatic activities punctuated by rapid and voluminous emplacement of extrusive and intrusive rocks in the rift led to the emplacement of a thick succession of volcanic rocks and mafic intrusions in Laurentia’s interior. Syn- to post-rift thermal subsidence led to deposition of clastic sedimentary rocks on top of the igneous rocks.

\begin{figure}
    \centering
    \includegraphics{chapters_overview.pdf}
    \caption{Caption}
    \label{fig:overview}
\end{figure}

The Midcontinent Rift eventually ceased and failed to split the continent into two. Far-field compressional forces associated with the onset and development of the Grenvillian orogeny along the eastern margin of Laurentia led to cessation and subsequent inversion of the rift \cite{Cannon1993a, Swanson-Hysell2019a}. In the southern Lake Superior region, Midcontinent Rift volcanic and sedimentary rocks were uplifted along with Paleoproterozoic and Archean lithologies via thrust faults, forming the crustal-scale Montreal River monocline \cite{Cannon1993a}. The resultant topography led to the deposition of the early Neoproterozoic Jacobsville Formation \cite{Hamblin1958a, Kalliokoski1982a, Hodgin2022a}, which overlies an angular unconformity that developed on lithologies that were exhumed through this earlier episode of contractional deformation associated with Grenvillian orogenesis. 

The intracontinental nature of the rift and the its cessation led to large amount of rift rocks being preserved in the interior of Laurentia far from continental margins. Rocks in the rift were subsequently experienced mild burial and metamorphism history. Rb-Sr dates from the uplifted basement rocks in southern Lake Superior region show that the area has not been heated to above 300 \textdegree C since ca. 1050 Ma \cite{Cannon1993a}. $^{10}$Be data show that some surface bedrock exposures have only recently been near the surface due to Pleistocene glacial and recent fluvial erosion \cite<e.g.>{Ullman2015a}. 

The rich record of well-preserved Midcontinent Rift rocks provide a wealth of opportunities for characterizing the magmatic history of Laurentia at the time. Geochronology and geochemistry data have been used to divide magmatic activity in the rift into four stages based on interpreted changes in relative magmatic volume and the nature of magmatism: early ($\sim$1109–1104 Ma), latent ($\sim$1104–1098 Ma), main ($\sim$1098–1090 Ma) and late ($\sim$1090–1083 Ma) \cite{Vervoort2007a, Heaman2007a, Miller2013a}. Advances in the method of chemical abrasion-isotope dilution-thermal ionization mass spectroscopy (CA-ID-TIMS) and its application in obtaining high-precision zircon U-Pb ages, together with the development of high-quality paleomagnetic data from the extrusive and intrusive rocks, have revealed distinct rapid and voluminous episodes of magmatism in the rift during the overall protracted period of active magmatism \cite{Swanson-Hysell2019a, Swanson-Hysell2021a, Zhang2021b}. In particular, a large igneous province, consisting of the massive Duluth Complex and its associated $\sim$8 km thick extrusive lava flows of the North Shore Volcanic Group, formed within 500 Kyr ca. 1096 Ma \cite{Swanson-Hysell2021a}. Stratigraphically above the Duluth Complex, the ca. 1092 Ma mafic Beaver Bay Complex represents another rapid and voluminous pulse of magmatism. It features the emplacement of the hypabyssal intrusion of the Beaver River diabase that has wide magma conduits containing giant anorthosite xenoliths that can be as much as 300 meters in width. Unlike the Duluth Complex and the North Shore Volcanic Group, whose magmatic linkage is supported by the exposed stratigraphic relationships and geochronologic data, there is no direct exposure of volcanic rocks that overlie conformably on top of the Beaver River diabase. Previously on the basis of geochemical data and petrographic observations, a hypothesis was put forward that the Greenstone Flow of the Portage Lake Volcanics that outcrop in the upper Keweenaw Peninsula in Michigan could be the surface expression of the Beaver River diabase. In chapter 1, I test this hypothesis by using paleomagnetic and geochronologic data to investigate the synchroneity of the emplacement of the intrusive and extrusive rocks. The data support that the Beaver River diabase was the feeder system for the $\sim$400 meters thick Greenstone Flow which ranks one of the largest lava flows known on Earth. In chapter 2, I utilize the exceptionally well-preserved anorthosite xenoliths hosted in the Beaver River diabase to shed new light on the evolution of Earth's geodynamo.

While rapid and voluminous mafic magmatism occurred in the Midcontinent Rift, improved geochronologic and paleomagnetic data show that distinct episodes of mafic magmatism occurred in southwestern Laurentia ca. 1098 Ma and ca. 1082 Ma \cite{Mohr2024a}. During a period of $<$0.25 Myr ca. 1098 Ma, thick mafic diabase sills and dikes intruded through southwestern basement rocks into the Mesoproterozoic sedimentary rocks including the Crystal Spring Formation in the Death Valley, Unkar Group sedimentary rocks in the Grand Canyon, and the Apache Group sedimentary rocks in central Arizona \cite{Bright2014a}. The tempo and extent of this episode of mafic magmatism and juvenile geochemical signature of the rocks are consistent with there being a southwestern large igneous province driven by a mantle plume. That the emplacement of the voluminous intrusions in the southwest occurred 2 Myr prior to the Duluth Complex led \cite{Mohr2024a} to invoke a tectonic model, where the mantle plume first arrived ca. 1098 Ma in southwestern Laurentia and laterally transported toward the readily thinned crust of the Midcontinent Rift. The arrival of the plume material in the rift postdates the latent magmatic stage ($\sim$1104–1098 Ma) and led to a replenishment of mafic magma and heat supply in the rift, which eventually drove the emplacement of the ca. 1096 Ma Duluth Complex and the associated lava flows. In chapter 3 I develop new paleomagnetic data from southwestern Laurentia and update the extent of the southwestern large igneous province. 

The Grenvillian orogenesis is a major event leading up to the assembly of the supercontinent Rodinia with Laurentia at its center \cite{Swanson-Hysell2023a}. Magmatism waned in the Midcontinent Rift and Laurentia's plate speed slowed down as the continental-scale collision commenced along Laurentia's eastern margin \cite{Swanson-Hysell2019a}. Following this time, there has been a lack of magmatism in Lauretia and thus a lack of well-dated paleomagnetic data that constrains global paleogeography until Rodinia began to rift apart $\sim$300 Myr later \cite{Eyster2019a}. The deposition of the post-rift Jacobsville Formation has been dated to be ca. 990 Ma and thus provides an opportunity to constrain the position of Rodinia in the earliest Neoproterozoic. In chapter 4 I develop new paleomagnetic data from the Jacobsville Formation in a detailed stratigraphic context. 

% A fruitful history of the development of paleomagnetic and high-precision geochronology data has helped advance the understanding of late Mesoproterozoic plate tectonics and Earth interior processes. Syntheses of paleomagnetic directional data in detailed volcanostratigraphic context, and pairing such data with high-precision geochronology data has found that Earth's magnetic field at the time was dominantly dipolar with symmetric reversals \cite{Swanson-Hysell2014a}, had a paleosecular variation patter similar to that today \cite{Tauxe2009a}, and likely developed superchrons \cite{Driscoll2016b}. The rapid pole progression from ca. 1110 Ma to ca. 1080 Ma shown by the Keweenawan Track indicate a period of rapid plate tectonic motion of Laurentia leading up to the Grenville continental collisional orogenesis and the assembly of the supercontinent Rodinia \cite{Swanson-Hysell2019a, Swanson-Hysell2023a, Rose2022a}. The new paleomagnetic directional data from the Midcontinent Rift region and the southwestern Laurentia can further constrain the apparent polar wander path of Laurentia in the late Mesoproterozoic. I also develop new paleomagnetic intensity data from anorthosite xenoliths in the Midcontinent Rift to probe the strength of the geodynamo ca. 1092 Ma. 

\subsection{The GAD hypothesis in the late Mesoproterozoic}

Earth's magnetic field today is dominated by the dipole component \cite{Alken2021a}. It is the working hypothesis that the surface expression of Earth’s geomagnetic field in deep time averages out to a geocentric axial dipole (GAD) whose poles coincides with the geographic poles that grants power to paleomagnetic directional observations. In the GAD model, an observation of an inclination of magnetization (I) uniquely corresponds to a latitude value ($\lambda$). This direction-pole relationship is expressed in the ``dipole equation" 
\begin{equation}
    tan(I) = 2 tan(\lambda)
\label{direction_eq}
\end{equation} The GAD hypothesis also empowers paleomagnetic field intensity observations in that there is a unique correspondence between an observed field strength at any latitude and the axial dipole moment, which is mathematically expressed as 
\begin{equation}
    B = \frac{\sqrt{1+3cos^2\theta}}{r^3}
\label{intensity_eq}
\end{equation} Where B is the observed field intensity, $\theta$ is colatitude where $\theta=90-\lambda$, and r is Earth's radius. 

% add inclination-latitude figure and geomagnetic field-dipole moment strength paired plot
\begin{figure}
    \centering
    \includegraphics{}
    \caption{Caption}
    \label{fig:dipole_equations}
\end{figure}

Effective ways of testing the robustness of the GAD hypothesis through time include comparing observed latitudinal dependence of paleomagnetic inclination data and intensity data with those predicted by the dipole equations. In addition, the symmetry of paleomagnetic reversals also provides insights into the dipolarity of the field, in that a geomagnetic field with a dominant non-dipolar component that does not reverse and a reversing dipolar component with minor contribution would have asymmetric reversals. It has been demonstrated that the GAD hypothesis is robust for the past 10,000 years on the basis of the magnetization of global compilations of archeological sites and sediments \cite<e.g.>{McElhinny1996a}. For the past 5 million years, compilations of paleomagnetic data are best described with a GAD-dominated field with minor persistent contributions from non-dipole components (1-5 \%; \cite{Johnson1997a, Tauxe2005a, Valet2011b}). Statistical analyses of compilations of the relatively abundant paleomagnetic directional and intensity data further back in time show that Earth's magnetic field has been compatible with being GAD-dominated throughout the Phanerozoic \cite{Evans1976a, Lhuillier2023a}. The use of paleomagnetic data under the GAD assumption lies at the core of reconstructing past plate tectonic configurations \cite{Creer1957a, Runcorn1962a, Irving1977a, Besse2002a, Torsvik2012a}. The GAD model gains support in the Phanerozoic in that the reconstructed plate motions are smooth with rates typically $<$20 cm/y \cite{Torsvik2012a}. Such results are independently verified by marine magnetic anomalies and hotspot tracks since the mid-Mesozoic \cite{Doubrovine2012a, Müller1993a}.

However, the Precambrian paleomagnetic observations challenge the uniformitariansm of a dipolar geomagnetic field through deeper time. There is increasing evidence that suggest Earth's magnetic field in the Ediacaran may not be GAD-dominated. Anomalously rapid changes in paleomagnetic field directions have been found in rocks from a variety of localities \cite<e.g.>{Abrajevitch2010a, Meert2014a, Halls2015a}. Magnetostratigraphy data developed by \cite{Kodama2021a} show the Johnnie Formation records anomalously frequent reversals ($\sim$13 per Myr) in the Ediacaran. A series of low paleointensity values have been obtained primarily using single silicate crystals (i.e. dominantly monomineralic crystal aggregates) from Ediacaran rocks \cite{Bono2019a, Thallner2021a, Thallner2021b}. All these data have invoked a hypothesis that Earth's magnetic field was not GAD-dominated in the Ediacaran Period. Some studies have posited that these data indicate a time when the geodynamo reached its minimum prior to a recovery brought by the onset of inner core nucleation \cite<e.g.>{Bono2019a}. Today, the characteristics of the geodynamo in the Ediacaran remain enigmatic, and the mechanism for the anomalous directional and intensity data remains debated \cite{Domeier2023a}.

In the Proterozoic, it had also been previously hypothesized that non-dipole components dominated the geomagnetic field. Prior to the high-precision geochronology data and detailed paleomagnetic data was developed within a volcanostratigraphic context became available, the apparent asymmetry in paleomagnetic field reversal records in Midcontinent Rift rocks led \cite{Halls1982a, Pesonen1981a} to interpret there being a significant non-GAD field component in the late Mesoproterozoic. However, high-resolution paleomagnetic data that span three geomagnetic field reversals in a more detailed stratigraphic context show each reversal was symmetric \cite{Swanson-Hysell2009a}. Since that work, more sophisticated statistical analyses have further clarified that the paleomagnetic pole progression recorded by rocks of the Midcontinent Rift represent rapid plate tectonic motions and does not require invoking a significant non-GAD geomagnetic field component or a significant true polar wander event \cite{Swanson-Hysell2019a, Rose2022a}. 

Additional statistical analyses by \cite{Tauxe2009a} show that the shape of the distribution of the paleomagnetic directional data recorded by lava flows of the North Shore Volcanic Group is consistent with that of the past 5 million years (model TK03 of \cite{Tauxe2004b}). My collaborative work with \cite{Pierce2022a} also show that correcting the inclination shallowing of syn-rift hematite-bearing sedimentary rocks of the Cut Face Creek Sandstone to be conformable with the field pattern of the TK03 model results in a mean direction that agrees with that recorded by the volcanic rocks that bracket the sedimentary rocks. These analyses support the interpretation that the geomagnetic field at Midcontinent Rift time was consistent with being dipole-dominated and had a secular variation patter similar to that of recent geologic times. The analyses of \cite{Gong2023a} show that paleomagnetic directional observations across Laurentia in the late Mesoproterozoic can be better fit with a dipole than a quadrupole configuration. Overall, the enriching paleomagnetic directional and intensity data in the Proterozoic have been adding increasing support for there being a GAD-dominated geomagnetic field through the Proterozoic Eon \cite{Veikkolainen2014a, Salminen2017a, Veikkolainen2021a, Gong2023a}. 

With confidence in the GAD-dominated field in the Mesoproterozoic, we can utilize paleomagnetic directional data to explore temporal associations between rock units when radiometric chronological constraints are not available. In chapter 1 I develop new paleomagnetic data from the ca. 1092 Ma Beaver River diabase and compare their pole position to that of the Greenstone Flow of the Portage Lake Volcanics. The agreement between the pole positions between the rift extrusive and intrusive rocks indicate close temporal linkage and leads way to a magmatic linkage between the two units. In chapter 3, I develop new paleomagnetic data from mafic intrusive and extrusive rocks in southwestern Laurentia and investigate the extent of the southwestern large igneous province, and its temporal and magmatic relationship with the Duluth Complex. These new data add to the existing paleomagnetic database for Laurentia for the reconstruction of the Keweenawan Track---the the late Mesoproterozoic apparent polar wander path of Laurentia which is central to the reconstruction of global paleogeography leading up to the assembly of the Rodinia supercontinent. 

\subsection{Using detrital remanent magnetization in hematite-bearing sedimentary rocks for paleogeographic reconstructions}

Sedimentary rocks are important archives of paleomagnetic records that provide major contributions to global paleogeography reconstructions \cite<e.g.>{Torsvik2012a, Domeier2012a, Vaes2023a}. While igneous rocks acquire thermal remanent magnetizations during cooling through blocking temperatures of ferromagnetic minerals, sedimentary rocks acquire detrital remanence magnetizations as a result of preferential alignment of detrital ferromagnetic particles with local magnetic field during deposition. In the Midcontinent Rift, clastic sedimentary rocks of the Oronto Group deposited during post-rift thermal subsidence following the bulk of rift magmatic activity provide paleomagnetic poles that extend the Keweenawan Track to ca. 1070 Ma \cite{Henry1977a, Slotznick2023a}. 

However, the accuracy of paleomagnetic directions recorded by the detrital remanent magnetization of sedimentary rocks has long been recognized as problematic due to the issue of inclination shallowing \cite{King1955a, Kodama2012a, Tauxe1984a, Van-Andel1966a}. The rotation of ferromagnetic grains during deposition and compaction can result in the acquisition of a detrital remanence that is biased shallow relative to the local geomagnetic field in which it was acquired \cite{Tauxe2005a}. Mathematically, the effect of inclination shallowing can be described as 
\begin{equation*}
\textup{tan}(I_o) = f\textup{tan}(I_f)
\end{equation*}
where $I_o$ represents the observed inclination of the specimen magnetization and $I_f$ represents the inclination of the field in which the magnetization was acquired. If uncorrected, shallower inclinations obtained from sedimentary rocks can potentially result in erroneously low estimates of paleolatitudes, biasing the interpreted past positions of continents and hindering plate reconstructions.

Experimental methods such as magnetic fabric analyses \cite<e.g.>{Kodama1995a, Bilardello2010a, Bilardello2010b, Bilardello2010c} as well as statistical approaches \cite<e.g.>{Tauxe2004b} have been developed for correcting inclination shallowing in sedimentary records. However, there had been limited efforts in reporting uncertainties associated with the amount of shallowing, let alone propagating such uncertainties into paleogeographic reconstructions. I developed a new method of representing the uncertainties associated with inclination shallowing by using a spherical bivariate Kent distribution to represent uncertainty associated with paleomagnetic mean pole position through collaboration with an undergraduate honors thesis project which is published as \cite{Pierce2022a}. This method has had successful applications in paleogeography reconstructions \cite{Slotznick2023a, Vaes2023a}. 

% add inclination shallowing cartoon
\begin{figure}
    \centering
    \includegraphics{}
    \caption{Caption}
    \label{fig:inclination_shallowing_cartoon}
\end{figure}

% add Cut Face figure
\begin{figure}
    \centering
    \includegraphics{}
    \caption{Caption}
    \label{fig:inclination_shallowing_correction}
\end{figure}

In chapter 4, I develop new paleomagnetic data from hematite-bearing sedimentary rocks of the Jacobsville Formation which overlies unconformably on top of Midcontinent Rift. \cite{Hodgin2022a} developed geochronology data that constrain the timing of deposition of the Jacobsville Formation to be associated with compressional deformation of the Grenvillian orogenesis ca. 990 Ma. I use a Kent mean paleomagnetic pole position to represent the pole position and associated uncertainty recorded by the Jacobsville Formation. That pole is the first constraint from Laurentia's interior for the global paleogeography in the earliest Neoproterozoic. In that chapter I will show that the data together with the Keweenawan Track support the tectonic model where Laurentia traveled rapidly equatorward in the latest Mesoproterozoic and significantly slowed down following the onset of the Grenvillian orogenesis. 

\subsection{The global paleogeography in the earliest Neoproterozoic: progress and future directions}

The evolution of the configuration of ancient continents is foundational to our knowledge of global tectonics, geodynamics, and climate through Earth history. We know that in the late Proterozoic, ancient continents conjoined together to form the supercontinent Rodinia \cite{Swanson-Hysell2021c}. The position of Laurentia is particularly crucial given its central location in the supercontinent. Extensively studied paleomagnetic records of volcanics and intrusions associated with the Midcontinent Rift provide crucial constraints on the paleogeography of Laurentia in the late Mesoproterozoic Era \citep{Halls1982a, Swanson-Hysell2019a}. The resulting Keweenawan Track is a central record for reconstructing the assembly of the ancient supercontinent Rodinia \citep{Evans2021b}. The Keweenawan Track records rapid equatorward motion associated with the closure of the Unimos Ocean basin between Laurentia and conjugate continents leading up to the collisional Grenvillian orogeny that assembled Rodinia. The Grenvillian orogeny initiated ca. 1090 Ma with peak metamorphism of the Ottawan Stage of the orogeny occurring ca. 1050 Ma, but with protracted collisional orogenesis continuing through the 1020 to 980 Ma Rigolet Stage \citep{Rivers2008a, Rivers2012b, Swanson-Hysell2023a}. This later stage of the orogeny led to the formation of the Grenville Front as contractional deformation propagated further into the continent. 

However, following the ca. 1084 Ma cessation of magmatism in the Midcontinent Rift and coeval volcanism in southwestern Laurentia \citep{Bright2014a, Mohr2024a}, there was a $\sim$300 Myr quiescence in known intracratonic magmatic activity in Laurentia that lasted until the emplacement of the ca. 775 Ma Gunbarrel large igneous province \citep{Harlan2003a, Mackinder2019a, Swanson-Hysell2021c}. Although the sedimentary records of the Oronto Group in the Midcontinent Rift basin help constrain the paleogeography of Laurentia at the latest Mesoproterozoic, the lack of well-dated paleomagnetic constraints from Laurentia's interior in the Neoproterozoic leaves the paleogeography of Rodinia poorly constrained between its initial assembly and breakup. 

Additional paleomagnetic constraints for Rodinia following the onset of the Grenvillian orogenesis have been obtained from the metamorphosed rocks of the orogen \cite<e.g.>{Irving1972a, Berger1979a}. Metamorphic thermobarometry studies have found that regional metamorphism of the Grenville orogen reached up to Granulite facies with peak metamorphism temperatures estimated to be $\sim$750\textdegree C \cite{}, possibly even higher \cite{Shinevar2021a, Metzger2021a}. Such temperatures are above the Curie temperature of magnetite ($\sim$580\textdegree C) and the N\'eel temperature of hematite ($\sim$680\textdegree C). Therefore, rocks of the Grenville orogen would have had their remanent magnetization acquired during initial formation entirely erased by heating during peak metamorphism, and subsequently overwritten during cooling associated with exhumation long after peak metamorphic conditions \citep{McWilliams1975a}. Paleomagnetic poles from the Grenville orogen cluster in a position that is significantly south to the end of the Keweenawan Track, as well as the positions of later Neoproterozoic paleomagnetic poles. That Laurentia's apparent polar wander path loops down along the Keweenawan Track to the poles from the Grenville orogen and then back up again has led to them being referred to as the ``Grenville Loop'' \citep{Berger1979a}. Given that Laurentia had a central position within Rodinia, reconstructions of the supercontinent are heavily reliant on these pole positions and the ages that are assigned to them both in terms of the supercontinent's absolute position and the relative configuration between different cratons and Laurentia. 

Determining the age of the the timing of remanence acquisition in rocks of the Grenville orogen requires constraining the cooling history of the orogen. Pioneering work pairing thermochronology with paleomagnetic data recognized the long duration of cooling within the orogen and the necessity of determining cooling curves to assign ages to paleomagnetic poles \citep{Berger1979a}. This work has subsequently been built upon through the development of additional $^{40}$Ar/$^{39}$Ar and U-Pb thermochronology data (e.g. \cite{Mezger1991a, Warnock2000a}) and the use of cooling curves interpreted from such data to interpret ages of paleomagnetic poles (e.g. \cite{Warnock2000a, Brown2012a}). As a result, high latitude Grenville Loop poles from the Haliburton intrusions in Ontario and the Adirondack Highlands have been assigned ages of 1015 Ma, 990 Ma, and 970 Ma. Albeit that these age assignments to Grenville Loop poles are based on rough estimates of the timing of magnetic remanence acquisition along simplified interpolation of cooling history paths, they have been incorporated into curated paleomagnetic database for paleogeography reconstructions \cite<e.g.>{Weil2003a, Evans2021a}. As is shown in Figure \ref{fig:pole_plot}, paleomagnetic poles of the Grenville Loop plot near Australia in present-day coordinates; forming arc distances of $\sim$50\textdegree\ from poles at the end of the ca. 1110 to 1070 Ma Keweenawan Track.

The now well-dated ca. 990 Ma paleomagnetic pole position of the Jacobsville Formation from Laurentia's interior in chapter 4 challenges the previous age assignments to the Grenville poles. In Figure \ref{fig:pole_plot}, the Grenville Loop poles also plot far away from the new Jacobsville pole. That ages assigned to the Grenville Loop are both the same and bracketing the age of the Jacobsville pole presents a conundrum given the very different positions they imply for Laurentia and associated continents in Rodinia (Fig. \ref{fig:paleogeography}). In chapter 4, robust field tests constrain the Jacobsville pole to be a primary detrital remanent magnetization that is chronostratigraphically well-constrained. We consider it to be a more reliable representation of the position of Laurentia ca. 990 Ma. However, there is a wealth of paleomagnetic data from the Grenville orogen that consistently reproduce the high-latitude pole position. Given that the Jacobsville pole is temporally associated with the last pulse of Grenvillian contractional deformation --- it was deposited in a synorogenic basin as the Grenville Front developed and propagated into the interior of Laurentia --- solutions for this discrepancy that call upon separation between Laurentia and the Grenville Province at the time are not feasible. The Grenville poles must be a representation of Laurentia's paleogeographic position, but the question is at what time? 

In previous Grenville paleomagnetic literature, the lack of reporting uncertainties associated with these age assignments is partly due to the lack of using quantitative methods to reconstruct probable thermal history paths based on the thermochronology data. The fact that measurement level data and specimen-specific details such as grain sizes are not available for the historic thermochronology literature makes it difficult to reproduce or improve those estimated ages. Additional uncertainty that was not fully taken into account in the previous literature comes from the cooling-rate dependence of magnetic remanence acquisition. In some contributions that have assigned ages to Grenville paleomagnetic poles, the Curie temperature of magnetite has been used as the temperature with which to assign an age to a magnetite magnetization from an interpreted cooling curve. While the Curie temperature is certainly relevant as above it the mineral is paramagnetic rather than ferromagnetic, blocking temperatures for an assemblage of grains will be below the Curie temperature. For populations of pre-existing magnetite, acquisition of magnetization occurs as a rock cools through temperatures where the thermal fluctuation energy is no longer sufficient to reset the magnetization of particles to align with the ambient field on a given timescale (the relaxation time). For rapidly cooled rocks such as lava flows, the cooling rate during emplacement and that during demagnetization in the lab are similar and the observed unblocking temperatures in the lab are close to the natural blocking temperatures during remanence acquisition (Fig. \ref{fig:slow_cooling}). In contrast to a rapidly cooled lava, magnetite-bearing rocks in the Grenville orogen acquired remanence during protracted cooling after peak metamorphism leading to a several orders of magnitude difference between the natural cooling rate and thermal demagnetization in the lab. This slow cooling leads to an expression of the cooling rate dependence where the temperatures at which a magnetization is acquired are lower than the temperatures at which it is removed in the lab (Fig. \ref{fig:slow_cooling}; \cite{Pullaiah1975b, Halgedahl1980a, Dodson1980a}). The example shown in Figure \ref{fig:slow_cooling}B compares the natural blocking temperature to the lab unblocking temperature for a cooling rate of 6\textdegree C/Myr using the framework of \cite{Dodson1980a}. In this case, magnetic remanence acquired at a temperature of 430\textdegree C would unblock in the lab at 500\textdegree C. Estimating the age of remanence acquisition in conjunction with cooling trajectories necessitates developing a framework that implements these relationships. These considerations further highlight how important it is to precisely determine the unblocking temperatures of magnetite-held remanence associated with a paleomagnetic direction. Previous efforts to assign ages based on cooling curves have typically picked single temperature values. 

I am interested in testing a hypothesis to explain the distinct pole positions between the Jacobsville pole and the Grenville Loop poles, which is that the poles from the Grenville orogen are younger than currently interpreted. Instead of the poles having remanence that was acquired while the Grenvillian orogeny was still ongoing (as is the case of the interpretation of the Haliburton pole; \cite{Warnock2000a}) or in the few 10s of millions of years following the Rigolet stage (as interpreted for the Adirondack poles; Fig. \ref{fig:thermal_history}), the magnetic remanence could instead have been acquired during exhumation well after the ca. 980 Ma cessation of contractional deformation. In this scenario, the migration of Rodinia to the higher latitude position represented by the Grenville Loop poles would have occurred further into the Neoproterozoic (Fig. \ref{fig:paleogeography}). Testing this hypothesis is of central importance to Neoproterozoic paleogeography as the ages associated with those Grenville Loop poles are crucial for constraining the motion of Laurentia at this time and the configuration between Laurentia and hypothesized conjugate continents. For example, in the case of Baltica it has been argued through comparison between paleomagnetic poles from the Grenville orogen and Sveconorwegian orogen that there were multiple oscillatory true polar wander rotations in the early Neoproterozoic \citep{Evans2015a, Gong2018b}. This interpretation is reliant on the assigned ages of the poles in both orogens. I plan to approach this question by developing high-precision U-Pb thermochronology data and new high-resolution paleomagnetic data. I will quantitatively construct thermal history paths and associated uncertainties using U-Pb apatite system as a thermochronometer which have closure temperatures similar to the blocking temperatures of titanomagnetite in Grenville rocks. I will pair these results with new high-resolution thermal demagnetization data such that the natural blocking temperature of the Grenville rocks can be better constrained based on the unblocking spectra obtained in the lab, taking the cooling-rate effect into consideration. 


\subsection{Probing the evolution of Earth's interior via paleointensity observations}

Earth's magnetic field is powered by convective flow of liquid iron-alloy in Earth's outer core. At present day, the geodynamo is collectively driven by heat flow across the core-mantle boundary and from the crystallization of the solid inner core from the liquid outer core which provides latent heat and compositional buoyancy due to the exclusion of light elements \cite{Buffett2000a}. The global entropy balance for the convective core that relates the Ohmic dissipation to thermal and compositional convection show that the the compositional energy is more efficient in maintaining the current energy dissipation \cite{Labrosse2003a, Landeau2022a}. A question thus arise regarding whether Earth has had an inner core since the beginning of its formation, and if not, when did the inner core begin to nucleate. Theoretical calculations show large uncertainties associated with the predicted timing of the inner core nucleation due to the lack of constraints on the thermal conductivity of the core \cite{Gubbins2004a, Konopkova2016a, Pozzo2012a, Ohta2016a}. Such uncertainties remain until further constraints on the core composition and agreements on experimental results of thermal conductivity values of iron-alloy become available in the literature. 

Paleomagnetic observations provide a unique tool that can provide constraints on the thermal evolution of Earth’s core. Based on paleomagnetic directional and intensity data, we know that an active dynamo  has existed since the 3.5 Ga \cite{Selkin2007a, Biggin2011a, Tarduno2014a, Brenner2020a, Brenner2022a} and probably earlier \cite{Tarduno2015a}. Theory and modeling predict that the nucleation of the inner core would produce a notable increase in the efficiency of convection in the outer core such that a notable increase in the surface geomagnetic field intensity may be observed and thus be reflected in the paleomagnetic intensity records \cite<e.g.>{Labrosse2003a, Aubert2009a, Driscoll2016a, Landeau2022a}. 

Paleointensity experiments are used to reconstruct records of past geomagnetic field intensity from thermoremanent magnetizations based on the linear dependence of thermal remanence acquisition on the magnetic field present. Based on the single domain thermal remanence magnetization blocking theory of \cite{Neel1955a}, \cite{Butler1992a} deduced the following equation for the acquisition of thermal remanence magnetization (TRM) of a population of identical single-domain ferromagnetic grains when cooling in a constant magnetic field (H) from the blocking temperature ($T_B$) to ambient temperature ($T_0$):
\begin{equation}
    TRM [T_0] = N[T_B] v j_s [T_0] tanh(\frac{v j_s [T_B]H}{k T_B})
\label{PINT_eq_1}
\end{equation}
where $[T_0], [T_B]$ denote the temperature of the temperature-dependent parameters, $N[T_B]$ represents the number of single-domain grains per unit volume with blocking temperature $T_B$, $v$ is the volume of the single domain grains and $j_s$ is the saturation magnetization, k is the Boltzmann constant. This equation assumes the grains block in the remanence magnetization sharply when cooling through the blocking temperature $T_B$. Typically the term $\frac{v j_s [T_B]H}{k T_B}$ which measures the degree of alignment of the grains at the blocking temperature has a value $<<1$. Therefore, equation \ref{PINT_eq_1} becomes 
\begin{equation}
    TRM [T_0] = AH
\label{PINT_eq_2}
\end{equation}

where A is a generalized proportionality constant. A rock sample that acquires a thermal remanent magnetization that follows this linear relationship with the magnetic field that it cools in would thus have a proportionality constant $A$ which depicts both $TRM_{paleo}=AH_{paleo}$ and $TRM_{lab}=AH_{lab}$. Therefore, the paleointensity can be obtained by eliminating the proportionality constant, A, and 
\begin{equation}
    H_{paleo} = \frac{TRM_{paleo}}{TRM_{lab}}
\label{PINT_eq_3}
\end{equation}

% add paleointensity example plot
\begin{figure}
    \centering
    \includegraphics{}
    \caption{Caption}
    \label{fig:VADM_compilation}
\end{figure}

With this theoretical basis, \cite{Thellier1959a} developed the experimental protocol later known as the Thellier double-heating paleointensity method that features step-wise removal of the natural thermal remanence magnetization ($TRM_{paleo}$) and imposition of new partial thermal remanence ($TRM_{lab}$) in a known lab field through heating and cooling. This technique is based upon Thellier’s laws, which state that partial thermal remanence (pTRM) must be additive, independent (partial remanence acquired between distinct temperature steps is distinct from one another) and reciprocal (the unblocking temperature is the same as the blocking temperature. The resultant comparison between a sample's capability of acquiring a TRM in the lab and its acquired TRM in nature is then illustrated in the paleointensity plot first introduced by \cite{Arai1963a} (Figure). In an ideal experiment with ideal magnetic recorders, the relationship between natural remanence lost and pTRM gained is linear, and the slope of the best-fit line to the data is proportional to the intensity of the ancient field. Using this slope, and multiplying by the known lab field, one can calculate the ancient magnetic field strength. The Thellier method has subsequently been improved such that sample behaviors that diverge from being ideal single-domain described in the theory can be better detected \cite<e.g.>{Coe1967b, Riisager2001a, Yu2003a, Yu2004a, Yu2006a}. Today, the ``in field-zero field-zero field-in field" (IZZI) Thellier paleointensity protocol combined with result filtering based on a series of statistical selection criteria is one of the most robust and widely used heating-based methods \cite{Yu2004a}. 

With the GAD assumption, Earth's (virtual) axial dipole moment can be calculated given known paleointensity and paleolatitude of a locality at a time (equation \ref{PINT_eq_1, PINT_eq_2}. There are compilations of paleointensity data that show Earth's (virtual) axial dipole moment through history (Figure; \cite<e.g.>{Veikkolainen2014b, Bono2022b}). Based on a selected compilation, \cite{Bono2019a} interpreted that the existing Precambrian paleointensity database shows a monotonic decay of the geomagnetic dipole moment throughout the Proterozoic. Together with their new data developed from silicate crystals of the ca. 565 Ma Sept-$\hat{I}$les mafic intrusions with the Thellier–Coe paleointensity method which were interpreted to show near-zero values, \cite{Bono2019a} hypothesized that a geodynamo primarily driven by waning thermal convection might have persistent through the Proterozoic until the nucleation of the inner core, which they interpreted to have happened in the Ediacaran. \cite{Bono2019a} further argued that their results are consistent with modeling results using an approach combining dynamo simulations and theoretical scaling relationships that predicted that progressive decay of the field’s dipole moment would be followed by a rapid increase in geomagnetic field intensity soon after the onset of inner core nucleation such that a minimum in dipole moment would occur just before inner core nucleation \cite{Driscoll2016a}. 

However, the geodynamo evolution model put forward by \cite{Bono2019a} was based on a sparse Precambrian paleointensity database and is challenged by more recent observations. Low paleointensity values similar to those in the Ediacaran have been observed in the Mesoproterozoic \cite{Lloyd2021b, Shcherbakova2022a, Shcherbakova2023a}, the Neoproterozoic \cite{Lloyd2021a}, the early-Cambrian \cite{Lloyd2022a}, and the Paleozoic \cite{Shcherbakova2021a}. These new data challenge the uniqueness of the Ediacaran weak geodynamo. In chapter 2, I develop new paleointensity data from the ca. 1092 Ma anorthosite xenoliths of the Beaver River diabase. The paleomagnetic directional data I developed in chapter 1 establish that the anorthosite xenoliths are stable remanence recorders that acquired remanence during cooling in the host diabase. I then used the ``IZZI" Thellier paleointensity protocol \cite{Yu2004a} to develop absolute paleointensity data from the same xenoliths and calculated the axial dipole moment of Earth's magnetic field. My data together with previous observations from rocks of the Midcontinent Rift show high paleointensity values ca. 1.1 Ga. Some of the xenoliths unequivocally show exceptionally high paleointensity values. Those high values require there being a strong dynamo in the late Mesoproterozoic, which does not fit in the interpreted Proterozoic monotonic decay trend hypothesized by \cite{Bono2019a}. 

Overall, the high paleointensity values of the Beaver River anorthosite xenoliths necessitate a strong late Mesoproterozoic geodynamo. The enriching paleomagnetic database for the Precambrian is revealing a more complex history if Earth's core than previously thought. The observational data from paleomagnetism are now stirring renewed interests in better understanding the core's material properties in deep-Earth conditions and in understanding the dynamic connections between the mantle and the core. 

% add VADM compilation plot highlighting the recent addition of data (say since 2019?) 
\begin{figure}
    \centering
    \includegraphics{}
    \caption{Caption}
    \label{fig:VADM_compilation}
\end{figure}


\subsection{Recovering deep-time paleointensity records from silicate-hosted Fe-oxides}

Successful recovery of deep-time paleomagnetic records is dependent on rocks maintaining primary magnetic remanence. Rocks are assemblages of fine-grained ferromagnetic particles with various stability that are dispersed within a matrix of diamagnetic and paramagnetic minerals. Not all ferromagnetic particles can maintain a magnetic memory through deep time. The characteristic time through which a ferromagnetic grain is capable of maintaining a magnetization before spontaneous relaxation is a function of temperature, external magneic field, composition, size, and shape. Theoretical derivations show that the most stable remanence carriers are single-domain particles with a narrow size range \cite{Butler1975a, Butler1992a}. For example, ensembles of magnetite grains with sizes ranging from a few tens of nanometers to up to one micron, and hematite grains with sizes up to 15 $\mu$m can maintain a stable magnetization for billions of years. Recent advances in micromagnetic modeling and fine-scale magnetic imaging show that stable ferromagnetic particles in geologic samples are likely to be more often in the single vortex state which behaves similarly to the single domain state \cite{Nagy2017a, Tauxe2020a, CortesOrtuno2022a}. 

Other ferromagnetic particles in rocks include multidomain grains (i.e. particles with multiple magnetic domains) and complex vortex states (or pseudo-single-domain grains) which have more complex magnetic behaviors \cite{Williams2010a}. While some of them may be capable of maintaining primary paleomagnetic directional records through deep time, others are more prone to having their magnetizations be overwritten by heating or prolonged immersion in Earth's magnetic field postdating the acquisition of the primary remanence. Results of such processes typically manifest in rock samples as superposition of secondary remanence components on top of variably preserved primary remanence. 

While isolation of primary paleomagnetic directional data from secondary overprints is possible via step-wise thermal demagnetization experiments and vector subtractions, the existence of multidomain and complex vortex state grains complicate the determination of paleointensity records as their remanence acquisition and removal behaviors deviate from the Thellier laws. In addition, alteration of rocks in nature and in the lab can also complicate paleomagnetic records by changing the magnetic mineralogy. This could result in the destruction of original minerals and the formation of new magnetic minerals associated with chemical alteration during heating in the lab. Such mineralogical alterations are detrimental to paleointensity experiments, since the acquired remanence during the in-field steps in the lab would be held by grains distinct from those that record the natural remanence. Due to such complexities, recovering deep time paleointensity records is usually more difficult than paleomagnetic directional records. 

It has been found that ferromagnetic grains enshrouded by silicate mineral hosts can be well-protected from alteration and be faithful paleointensity recorders \cite{Cottrell1999a, Cottrell2000a, Tarduno2005a, Tarduno2005a, Tarduno2006a, Cottrell2008a, Selkin2000a, Selkin2007a, Selkin2008a}. In particular, titanomagnetite grains hosted in plagioclase crystals are intriguing targets. Petrography, electron microscopy, and microprobe data have shown that minor amount ($<$1\% by weight) of iron commonly exist in plagioclase crystals which can be present as fine-scale magnetite needles \cite{Selkin2000a, Feinberg2005a, Feinberg2006a, Wenk2011a, Bian2021a}. Rock magnetic analyses show that many of the rocks have ferromagnetic mineral assemblages dominated by single-domian-like behaviors with stable remanence carrying abilities. In addition, the alteration of the plagioclase does not readily result in the formation of secondary iron oxides in contrast with Fe-silicate minerals such as olivine and pyroxene. 

In chapter 2, I develop new paleointensity data using the anorthosite xenoliths of the Beaver River diabase in the Midcontinent Rift to characterize Earth's magnetic field ca. 1092 Ma. Electron microscopy images show that plagioclase crystals within these anorthosite xenoliths can enclose small pyroxene crystals which have fine-scale exsolved titanomagnetite grains. These observations corresponds to rock magnetic data that a number of the xenoliths typically have a dominant population of stable single-domain-like particles. The failed intracontinental rift nature of the Midcontinent Rift results in the anorthosite xenoliths being distant from subsequent tectonic events along Laurentian margins such that the lithology and magnetization of the anorthosite xenoliths have been minimally altered following initial formation. 

% replot Butler 1975 a single domain diagram
\begin{figure}
    \centering
    \includegraphics{}
    \caption{Caption}
    \label{fig:Butler_single_domain}
\end{figure}


% show example vortex-state grain and stability field 
\begin{figure}
    \centering
    \includegraphics{}
    \caption{Caption}
    \label{fig:micromagnetic_model}
\end{figure}



\end{abstract}

% (This file is included by thesis.tex; you do not latex it by itself.)

\begin{abstract}

% \begin{center}
%     \textit{``An ancient time\\
%     When compass needles lost their way; \\
%     Finding North from South." \\}
%     \vspace{10pt}
%     S. W. Bogue
% \end{center}

Over 2000 years ago, one of humanity's earliest records of harnessing magnetism unfolded in ancient China, where lodestones (i.e. magnetic rocks) were used in telling directions in geomancy and divination practices. Compass made its way to Europe through the Silk Road, which granted European mariners the ability to tell their heading regardless of weather conditions or visibility constraints at sea. The improved navigation for mariners contributed to the Age of Discovery. Scientific advances through the centuries have made profound insights into the fundamental principles of electromagnetism. Today, our continued understanding and manipulation of the electromagnetic principles lie at the core of modern technology and life. 

It is because compass needles are made of magnetic materials that tend to align with Earth's magnetic field at surface which is dominantly dipolar and stable such that a compass can help navigate today. In the rock records, there are numerous microscopic compasses that are carrying ancient magnetic records through deep time. These microscopic compasses are magnetic minerals such as iron-titanium oxide minerals and iron sulfide minerals that occur rather ubiquitously in igneous, sedimentary, and metamorphic rocks. Through much of Earth's history, that a geodynamo that powered a strong magnetic field The study of paleomagnetism explores Earth history stories while being fundamentally rooted in the understanding of rock forming processes and the physics of magnetism.

In this dissertation, I find my applications of paleomagnetism in exploring the characteristics of magmatism in the Midcontinent Rift system in the Lake Superior region and in the southwestern Laurentia 1.1 billion years ago, shedding new lights in the strength of the geodynamo ca. 1092 million years ago, and furthering the development of global paleogeography through the end of the Mesoproterozoic and the earliest Neoproterozoic. 

\subsection{Late Mesoproterozoic magmatism in the Midcontinent Rift and the southwestern Laurentia large igneous province}

The ancient North American craton, Laurentia, first formed in the Paleoproterozoic Era when a series of collisional orogenies culminating the Trans-Hudson orogeny led to the amalgamation of Archean provinces \cite{Hoffman1988a, Whitmeyer2007a}. The craton continued to grow through the rest of the Paleoproterozoic and into the Mesoproterozoic via accretionary orogenesis along its margin. In the latest Mesoproterozoic (ca. 1110 to 1080 Ma) the large intracontinental Midcontinent Rift, which was colocated with a large igneous province (LIP) \cite{Swanson-Hysell2021a} led to extension within the Archean Superior province and adjacent Paleoproterozoic provinces to the south \cite{Cannon1992a}. Protracted magmatic activities punctuated by rapid and voluminous emplacement of extrusive and intrusive rocks in the rift led to the emplacement of a thick succession of volcanic rocks and mafic intrusions in Laurentia’s interior. Syn- to post-rift thermal subsidence led to the deposition of clastic sedimentary rocks on top of the igneous rocks.

The Midcontinent Rift eventually ceased and failed to split the continent into two. Far-field compressional forces associated with the onset and development of the Grenvillian orogeny along the eastern margin of Laurentia led to cessation and subsequent inversion of the rift \cite{Cannon1993a, Swanson-Hysell2019a}. In the southern Lake Superior region, Midcontinent Rift volcanic and sedimentary rocks were uplifted along with Paleoproterozoic and Archean lithologies via thrust faults, forming the crustal-scale Montreal River monocline \cite{Cannon1993a}. The resultant topography led to the deposition of the early Neoproterozoic Jacobsville Formation \cite{Hamblin1958a, Kalliokoski1982a, Hodgin2022a}, which overlies an angular unconformity that developed on lithologies that were exhumed through this earlier episode of contractional deformation associated with Grenvillian orogenesis. 

The intracontinental nature of the rift and the its cessation led to large amount of rift rocks being preserved in the interior of Laurentia far from continental margins. Rocks in the rift were subsequently experienced mild burial and metamorphism history. Rb-Sr dates from the uplifted basement rocks in southern Lake Superior region show that the area has not been heated to above 300 \textdegree C since ca. 1050 Ma \cite{Cannon1993a}. $^{10}$Be data show that some surface bedrock exposures have only recently been near the surface due to Pleistocene glacial and recent fluvial erosion \cite<e.g.>{Ullman2015a}. 

The rich record of well-preserved Midcontinent Rift rocks provide a wealth of opportunities for characterizing the magmatic history of Laurentia at the time. Geochronology and geochemistry data have been used to divide magmatic activity in the rift into four stages based on interpreted changes in relative magmatic volume and the nature of magmatism: early ($\sim$1109–1104 Ma), latent ($\sim$1104–1098 Ma), main ($\sim$1098–1090 Ma) and late ($\sim$1090–1083 Ma) \cite{Vervoort2007a, Heaman2007a, Miller2013a}. Advances in the method of chemical abrasion-isotope dilution-thermal ionization mass spectroscopy (CA-ID-TIMS) and its application in obtaining high-precision zircon U-Pb ages, together with the development of high-quality paleomagnetic data from the extrusive and intrusive rocks, have revealed distinct rapid and voluminous episodes of magmatism in the rift during the overall protracted period of active magmatism \cite{Swanson-Hysell2019a, Swanson-Hysell2021a, Zhang2021b}. In particular, a large igneous province, consisting of the massive Duluth Complex and its associated $\sim$8 km thick extrusive lava flows of the North Shore Volcanic Group, formed within 500 Kyr ca. 1096 Ma \cite{Swanson-Hysell2021a}. Stratigraphically above the Duluth Complex, the ca. 1092 Ma mafic Beaver Bay Complex represents another rapid and voluminous pulse of magmatism. It features the emplacement of the hypabyssal intrusion of the Beaver River diabase that has wide magma conduits containing giant anorthosite xenoliths that can be as much as 300 meters in width. Unlike the Duluth Complex and the North Shore Volcanic Group, whose magmatic linkage is supported by the exposed stratigraphic relationships and geochronologic data, there is no direct exposure of volcanic rocks that overlie conformably on top of the Beaver River diabase. Previously on the basis of geochemical data and petrographic observations, a hypothesis was put forward that the Greenstone Flow of the Portage Lake Volcanics that outcrop in the upper Keweenaw Peninsula in Michigan could be the surface expression of the Beaver River diabase. In chapter 1, I test this hypothesis by using paleomagnetic and geochronologic data to investigate the synchroneity of the emplacement of the intrusive and extrusive rocks. The data support that the Beaver River diabase was the feeder system for the $\sim$400 meters thick Greenstone Flow which ranks one of the largest lava flows known on Earth. In chapter 2, I utilize the exceptionally well-preserved anorthosite xenoliths hosted in the Beaver River diabase to shed new light on the evolution of Earth's geodynamo.

While rapid and voluminous mafic magmatism occurred in the Midcontinent Rift, improved geochronologic and paleomagnetic data show that distinct episodes of mafic magmatism occurred in southwestern Laurentia ca. 1098 Ma and ca. 1082 Ma \cite{Mohr2024a}. During a period of $<$0.25 Myr ca. 1098 Ma, thick mafic diabase sills and dikes intruded through southwestern basement rocks into the Mesoproterozoic sedimentary rocks including the Crystal Spring Formation in the Death Valley, Unkar Group sedimentary rocks in the Grand Canyon, and the Apache Group sedimentary rocks in central Arizona \cite{Bright2014a}. The tempo and extent of this episode of mafic magmatism and juvenile geochemical signature of the rocks are consistent with there being a southwestern large igneous province driven by a mantle plume. That the emplacement of the voluminous intrusions in the southwest occurred 2 Myr prior to the Duluth Complex led \cite{Mohr2024a} to invoke a tectonic model, where the mantle plume first arrived ca. 1098 Ma in southwestern Laurentia and laterally transported toward the readily thinned crust of the Midcontinent Rift. The arrival of the plume material in the rift postdates the latent magmatic stage ($\sim$1104–1098 Ma) and led to a replenishment of mafic magma and heat supply in the rift, which eventually drove the emplacement of the ca. 1096 Ma Duluth Complex and the associated lava flows. In chapter 3 I develop new paleomagnetic data from southwestern Laurentia and update the extent of the southwestern large igneous province. 

The Grenvillian orogenesis is a major event leading up to the assembly of the supercontinent Rodinia with Laurentia at its center \cite{Swanson-Hysell2023a}. Magmatism waned in the Midcontinent Rift and Laurentia's plate speed slowed down as the continental-scale collision commenced along Laurentia's eastern margin \cite{Swanson-Hysell2019a}. Following this time, there has been a lack of magmatism in Lauretia and thus a lack of well-dated paleomagnetic data that constrains global paleogeography until Rodinia began to rift apart $\sim$300 Myr later \cite{Eyster2019a}. The deposition of the post-rift Jacobsville Formation has been dated to be ca. 990 Ma and thus provides an opportunity to constrain the position of Rodinia in the earliest Neoproterozoic. In chapter 4 I develop new paleomagnetic data from the Jacobsville Formation in a detailed stratigraphic context. 

% A fruitful history of the development of paleomagnetic and high-precision geochronology data has helped advance the understanding of late Mesoproterozoic plate tectonics and Earth interior processes. Syntheses of paleomagnetic directional data in detailed volcanostratigraphic context, and pairing such data with high-precision geochronology data has found that Earth's magnetic field at the time was dominantly dipolar with symmetric reversals \cite{Swanson-Hysell2014a}, had a paleosecular variation patter similar to that today \cite{Tauxe2009a}, and likely developed superchrons \cite{Driscoll2016b}. The rapid pole progression from ca. 1110 Ma to ca. 1080 Ma shown by the Keweenawan Track indicate a period of rapid plate tectonic motion of Laurentia leading up to the Grenville continental collisional orogenesis and the assembly of the supercontinent Rodinia \cite{Swanson-Hysell2019a, Swanson-Hysell2023a, Rose2022a}. The new paleomagnetic directional data from the Midcontinent Rift region and the southwestern Laurentia can further constrain the apparent polar wander path of Laurentia in the late Mesoproterozoic. I also develop new paleomagnetic intensity data from anorthosite xenoliths in the Midcontinent Rift to probe the strength of the geodynamo ca. 1092 Ma. 

\subsection{The GAD hypothesis in the late Mesoproterozoic}

Earth's magnetic field today is dominated by the dipole component \cite{Alken2021a}. It is the working hypothesis that the surface expression of Earth’s geomagnetic field in deep time averages out to a geocentric axial dipole (GAD) whose poles coincides with the geographic poles that grants power to paleomagnetic directional observations. In the GAD model, an observation of an inclination of magnetization (I) uniquely corresponds to a latitude value ($\lambda$). This direction-pole relationship is expressed in the ``dipole equation" 
\begin{equation}
    tan(I) = 2 tan(\lambda)
\label{direction_eq}
\end{equation} The GAD hypothesis also empowers paleomagnetic field intensity observations in that there is a unique correspondence between an observed field strength at any latitude and the axial dipole moment, which is mathematically expressed as 
\begin{equation}
    B = \frac{\sqrt{1+3cos^2\theta}}{r^3}
\label{intensity_eq}
\end{equation} Where B is the observed field intensity, $\theta$ is colatitude where $\theta=90-\lambda$, and r is Earth's radius. 

% add inclination-latitude figure and geomagnetic field-dipole moment strength paired plot
\begin{figure}
    \centering
    \includegraphics{}
    \caption{Caption}
    \label{fig:dipole_equations}
\end{figure}

Effective ways of testing the robustness of the GAD hypothesis through time include comparing observed latitudinal dependence of paleomagnetic inclination data and intensity data with those predicted by the dipole equations. In addition, the symmetry of paleomagnetic reversals also provides insights into the dipolarity of the field, in that a geomagnetic field with a dominant non-dipolar component that does not reverse and a reversing dipolar component with minor contribution would have asymmetric reversals. It has been demonstrated that the GAD hypothesis is robust for the past 10,000 years on the basis of the magnetization of global compilations of archeological sites and sediments \cite<e.g.>{McElhinny1996a}. For the past 5 million years, compilations of paleomagnetic data are best described with a GAD-dominated field with minor persistent contributions from non-dipole components (1-5 \%; \cite{Johnson1997a, Tauxe2005a, Valet2011b}). Statistical analyses of compilations of the relatively abundant paleomagnetic directional and intensity data further back in time show that Earth's magnetic field has been compatible with being GAD-dominated throughout the Phanerozoic \cite{Evans1976a, Lhuillier2023a}. The use of paleomagnetic data under the GAD assumption lies at the core of reconstructing past plate tectonic configurations \cite{Creer1957a, Runcorn1962a, Irving1977a, Besse2002a, Torsvik2012a}. The GAD model gains support in the Phanerozoic in that the reconstructed plate motions are smooth with rates typically $<$20 cm/y \cite{Torsvik2012a}. Such results are independently verified by marine magnetic anomalies and hotspot tracks since the mid-Mesozoic \cite{Doubrovine2012a, Müller1993a}.

However, the Precambrian paleomagnetic observations challenge the uniformitariansm of a dipolar geomagnetic field through deeper time. There is increasing evidence that suggest Earth's magnetic field in the Ediacaran may not be GAD-dominated. Anomalously rapid changes in paleomagnetic field directions have been found in rocks from a variety of localities \cite<e.g.>{Abrajevitch2010a, Meert2014a, Halls2015a}. Magnetostratigraphy data developed by \cite{Kodama2021a} show the Johnnie Formation records anomalously frequent reversals ($\sim$13 per Myr) in the Ediacaran. A series of low paleointensity values have been obtained primarily using single silicate crystals (i.e. dominantly monomineralic crystal aggregates) from Ediacaran rocks \cite{Bono2019a, Thallner2021a, Thallner2021b}. All these data have invoked a hypothesis that Earth's magnetic field was not GAD-dominated in the Ediacaran Period. Some studies have posited that these data indicate a time when the geodynamo reached its minimum prior to a recovery brought by the onset of inner core nucleation \cite<e.g.>{Bono2019a}. Today, the characteristics of the geodynamo in the Ediacaran remain enigmatic, and the mechanism for the anomalous directional and intensity data remains debated \cite{Domeier2023a}.

In the Proterozoic, it had also been previously hypothesized that non-dipole components dominated the geomagnetic field. Prior to the high-precision geochronology data and detailed paleomagnetic data was developed within a volcanostratigraphic context became available, the apparent asymmetry in paleomagnetic field reversal records in Midcontinent Rift rocks led \cite{Halls1982a, Pesonen1981a} to interpret there being a significant non-GAD field component in the late Mesoproterozoic. However, high-resolution paleomagnetic data that span three geomagnetic field reversals in a more detailed stratigraphic context show each reversal was symmetric \cite{Swanson-Hysell2009a}. Since that work, more sophisticated statistical analyses have further clarified that the paleomagnetic pole progression recorded by rocks of the Midcontinent Rift represent rapid plate tectonic motions and does not require invoking a significant non-GAD geomagnetic field component or a significant true polar wander event \cite{Swanson-Hysell2019a, Rose2022a}. 

Additional statistical analyses by \cite{Tauxe2009a} show that the shape of the distribution of the paleomagnetic directional data recorded by lava flows of the North Shore Volcanic Group is consistent with that of the past 5 million years (model TK03 of \cite{Tauxe2004b}). My collaborative work with \cite{Pierce2022a} also show that correcting the inclination shallowing of syn-rift hematite-bearing sedimentary rocks of the Cut Face Creek Sandstone to be conformable with the field pattern of the TK03 model results in a mean direction that agrees with that recorded by the volcanic rocks that bracket the sedimentary rocks. These analyses support the interpretation that the geomagnetic field at Midcontinent Rift time was consistent with being dipole-dominated and had a secular variation patter similar to that of recent geologic times. The analyses of \cite{Gong2023a} show that paleomagnetic directional observations across Laurentia in the late Mesoproterozoic can be better fit with a dipole than a quadrupole configuration. Overall, the enriching paleomagnetic directional and intensity data in the Proterozoic have been adding increasing support for there being a GAD-dominated geomagnetic field through the Proterozoic Eon \cite{Veikkolainen2014a, Salminen2017a, Veikkolainen2021a, Gong2023a}. 

With confidence in the GAD-dominated field in the Mesoproterozoic, we can utilize paleomagnetic directional data to explore temporal associations between rock units when radiometric chronological constraints are not available. In chapter 1 I develop new paleomagnetic data from the ca. 1092 Ma Beaver River diabase and compare their pole position to that of the Greenstone Flow of the Portage Lake Volcanics. The agreement between the pole positions between the rift extrusive and intrusive rocks indicate close temporal linkage and leads way to a magmatic linkage between the two units. In chapter 3, I develop new paleomagnetic data from mafic intrusive and extrusive rocks in southwestern Laurentia and investigate the extent of the southwestern large igneous province, and its temporal and magmatic relationship with the Duluth Complex. These new data add to the existing paleomagnetic database for Laurentia for the reconstruction of the Keweenawan Track---the the late Mesoproterozoic apparent polar wander path of Laurentia which is central to the reconstruction of global paleogeography leading up to the assembly of the Rodinia supercontinent. 

\subsection{Probing the evolution of Earth's interior via paleointensity observations}

Earth's magnetic field is powered by convective flow of liquid iron-alloy in Earth's outer core. At present day, the geodynamo is collectively driven by heat flow across the core-mantle boundary and from the crystallization of the solid inner core from the liquid outer core which provides latent heat and compositional buoyancy due to the exclusion of light elements \cite{Buffett2000a}. The global entropy balance for the convective core that relates the Ohmic dissipation to thermal and compositional convection show that the the compositional energy is more efficient in maintaining the current energy dissipation \cite{Labrosse2003a, Landeau2022a}. A question thus arise regarding whether Earth has had an inner core since the beginning of its formation, and if not, when did the inner core begin to nucleate. Theoretical calculations show large uncertainties associated with the predicted timing of the inner core nucleation due to the lack of constraints on the thermal conductivity of the core \cite{Gubbins2004a, Konopkova2016a, Pozzo2012a, Ohta2016a}. Such uncertainties remain until further constraints on the core composition and agreements on experimental results of thermal conductivity values of iron-alloy become available in the literature. 

Paleomagnetic observations provide a unique tool that can provide constraints on the thermal evolution of Earth’s core. Based on paleomagnetic directional and intensity data, we know that an active dynamo  has existed since the 3.5 Ga \cite{Selkin2007a, Biggin2011a, Tarduno2014a, Brenner2020a, Brenner2022a} and probably earlier \cite{Tarduno2015a}. Theory and modeling predict that the nucleation of the inner core would produce a notable increase in the efficiency of convection in the outer core such that a notable increase in the surface geomagnetic field intensity may be observed and thus be reflected in the paleomagnetic intensity records \cite<e.g.>{Labrosse2003a, Aubert2009a, Driscoll2016a, Landeau2022a}. 

Paleointensity experiments are capable of reconstructing records of past geomagnetic field intensity from thermoremanent magnetizations based on the linear dependence of thermal remanence acquisition on the magnetic field present. Based on the single domain thermal remanence magnetization blocking theory of \cite{Neel1955a}, \cite{Butler1992a} deduced the following equation for the acquisition of thermal remanence magnetization (TRM) of a population of identical single-domain ferromagnetic grains when cooling in a constant magnetic field (H) from the blocking temperature ($T_B$) to ambient temperature ($T_0$):
\begin{equation}
    TRM [T_0] = N[T_B] v j_s [T_0] tanh(\frac{v j_s [T_B]H}{k T_B})
\label{PINT_eq_1}
\end{equation}
where $[T_0], [T_B]$ denote the temperature of the temperature-dependent parameters, $N[T_B]$ represents the number of single-domain grains per unit volume with blocking temperature $T_B$, $v$ is the volume of the single domain grains and $j_s$ is the saturation magnetization, k is the Boltzmann constant. This equation assumes the grains block in the remanence magnetization sharply when cooling through the blocking temperature $T_B$. Typically the term $\frac{v j_s [T_B]H}{k T_B}$ which measures the degree of alignment of the grains at the blocking temperature has a value $<<1$. Therefore, equation \ref{PINT_eq_1} becomes 
\begin{equation}
    TRM [T_0] = AH
\label{PINT_eq_2}
\end{equation}

where A is a generalized proportionality constant. A rock sample that acquires a thermal remanent magnetization that follows this linear relationship with the magnetic field that it cools in would thus have a proportionality constant $A$ which depicts that $TRM_{paleo}=AH_{paleo}$ and $TRM_{lab}=AH_{lab}$. Therefore, the paleointensity can be obtained by eliminating the proportionality constant, A, and 
\begin{equation}
    H_{paleo} = \frac{TRM_{paleo}}{TRM_{lab}}
\label{PINT_eq_3}
\end{equation}

% add paleointensity example plot
\begin{figure}
    \centering
    \includegraphics{}
    \caption{Caption}
    \label{fig:VADM_compilation}
\end{figure}

With this theoretical basis, \cite{Thellier1959a} developed the experimental protocol later known as the Thellier paleointensity method that features step-wise removal of the natural thermal remanence magnetization ($TRM_{paleo}$) and imposition of new thermal remanence ($TRM_{lab}$) in a known lab field through heating and cooling. The resultant comparison between a sample's capability of acquiring a TRM in the lab and its acquired TRM in nature is then illustrated in the paleointensity plot first introduced by \cite{Arai1963a} (Figure). The Thellier method has subsequently been improved such that sample behaviors that diverge from being ideal single-domain described in the theory can be better detected \cite<e.g.>{Coe1967b, Yu2003a, Yu2004a, Yu2006a}. Today, the ``in field-zero field-zero field-in field" (IZZI) Thellier paleointensity protocol combined with result filtering based on a series of statistical selection criteria is one of the most robust and widely used heating-based methods \cite{Yu2004a}. 

With the GAD assumption, Earth's (virtual) axial dipole moment can be calculated given known paleointensity and paleolatitude of a locality at a time (equation \ref{PINT_eq_1, PINT_eq_2}. There are compilations of paleointensity data that show Earth's (virtual) axial dipole moment through history (Figure; \cite<e.g.>{Veikkolainen2014b, Bono2022b}). Based on a selected compilation, \cite{Bono2019a} interpreted that the existing Precambrian paleointensity database shows a monotonic decay of the geomagnetic dipole moment throughout the Proterozoic. Together with their new data developed from silicate crystals of the ca. 565 Ma Sept-$\hat{I}$les mafic intrusions with the Thellier–Coe paleointensity method which were interpreted to show near-zero values, \cite{Bono2019a} hypothesized that a geodynamo primarily driven by waning thermal convection might have persistent through the Proterozoic until the nucleation of the inner core, which they interpreted to have happened in the Ediacaran. \cite{Bono2019a} further argued that their results are consistent with modeling results using an approach combining dynamo simulations and theoretical scaling relationships that predicted that progressive decay of the field’s dipole moment would be followed by a rapid increase in geomagnetic field intensity soon after the onset of inner core nucleation such that a minimum in dipole moment would occur just before inner core nucleation \cite{Driscoll2016a}. 

However, the geodynamo evolution model put forward by \cite{Bono2019a} was based on a sparse Precambrian paleointensity database and is challenged by more recent observations. Low paleointensity values similar to those in the Ediacaran have been observed in the Mesoproterozoic \cite{Lloyd2021b, Shcherbakova2022a, Shcherbakova2023a}, the Neoproterozoic \cite{Lloyd2021a}, the early-Cambrian \cite{Lloyd2022a}, and the Paleozoic \cite{Shcherbakova2021a}. These new data challenge the uniqueness of the Ediacaran weak geodynamo. In chapter 2, I develop new paleointensity data from the ca. 1092 Ma anorthosite xenoliths of the Beaver River diabase. The paleomagnetic directional data I developed in chapter 1 establish that the anorthosite xenoliths are stable remanence recorders that acquired remanence during cooling in the host diabase. I then used the ``IZZI" Thellier paleointensity protocol \cite{Yu2004a} to develop absolute paleointensity data from the same xenoliths and calculated the axial dipole moment of Earth's magnetic field. My data together with previous observations from rocks of the Midcontinent Rift show high paleointensity values ca. 1.1 Ga. Some of the xenoliths unequivocally show exceptionally high paleointensity values. Those high values require there being a strong dynamo in the late Mesoproterozoic, which does not fit in the interpreted Proterozoic monotonic decay trend hypothesized by \cite{Bono2019a}. 

Overall, the high paleointensity values of the Beaver River anorthosite xenoliths necessitate a strong late Mesoproterozoic geodynamo. The enriching paleomagnetic database for the Precambrian is revealing a more complex history if Earth's core than previously thought. The observational data from paleomagnetism are now stirring renewed interests in better understanding the core's material properties in deep-Earth conditions and in understanding the dynamic connections between the mantle and the core. 

% add VADM compilation plot highlighting the recent addition of data (say since 2019?) 
\begin{figure}
    \centering
    \includegraphics{}
    \caption{Caption}
    \label{fig:VADM_compilation}
\end{figure}


\subsection{Using silicate-hosted Fe-oxides for paleomagnetism in deep time}

Rocks are assemblages of fine-grained ferromagnetic particles with various stability that are dispersed within a matrix of diamagnetic and paramagnetic minerals. Not all ferromagnetic particles can maintain a magnetic memory through deep time. The characteristic time through which a ferromagnetic grain is capable of maintaining a certain magnetization is a function of composition, size, and shape. Theoretical derivations show that the most stable remanence carriers are single-domain particles with a narrow size range \cite{Butler1975a, Butler1992a}. For example, ensembles of magnetite grains with sizes ranging from a few tens of nanometers to up to one micron, and hematite grains with sizes up to 15 $\mu$m can maintain a stable magnetization for billions of years. Recent advances in micromagnetic modeling and fine-scale magnetic imaging show that stable ferromagnetic particles in geologic samples are likely to be more often in the vortex state instead of the single domain state \cite{Nagy2017a, Tauxe2020a, CortesOrtuno2022a}. Nevertheless, they also require small particle sizes to maintain a stable vortex state.

Magnetization held by non-ideal particles is more prone to be overwritten during post-formation processes such as heating. Secondary overprints are commonly observed in rock records associated with heating events that postdate the acquisition of the primary remanence. Viscous overprints are another common observations in rock records due the acquisition of recent magnetic remanence due to prolonged exposure in the geomagnetic field during recent geologic times. 

While the non-ideal remanence carriers may be capable of maintaining paleomagnetic directional records through deep time, they are detrimental to paleointensity experiments. Multidomain grains are shown to have asymmetric unblocking and blocking behaviors with additional complex dependence on the applied external field direction. Secondary overprints acquired at times distinct from the timing of primary remanence does not provide information on the geodynamo at time of rock formation and can scew the interpretation. 

In addition to magnetization overprints, alteration of rocks in nature and in the lab can also complicate paleomagnetic records by inducing changes to magnetic mineralogy. This results in the destruction of original minerals and the formation of new magnetic minerals associated with chemical alteration during heating in the lab. Such mineralogical alterations are detrimental to paleointensity experiments, since the acquired remanence during the in-field steps in the lab would be held by newly formed grains that are distinct from those that record the natural remanence. These behaviors thus violate the theoretical assumption of linear acquisition of thermal remanent magnetization. 

Therefore, determinations of absolute values of ancient geomagnetic field strength with high-fidelity rely on igneous rocks with ideal ferromagnetic particles that acquire thermal remanent magnetizations as they cool. These magnetizations also need to be unmodified by subsequent heating or chemical alteration in order to maintain the record of the ancient geomagnetic field from the time of cooling. Such strict requirements on the lithology and geologic history commonly make obtaining paleointensity data more challenging than measuring directional records. The Beaver River anorthosite xenoliths of the Midcontinent Rift stand out as intriguing targets given their mineralogy and tectonic setting. The intracontinental rift nature of the Midcontinent Rift results in the anorthosite xenoliths being distant from subsequent tectonic events along Laurentian margins that can drive alteration through heat and fluid flow. Additionally, extension ceased in the rift prior to lithospheric separation, preserving volcanic, intrusive, and sedimentary rocks of the rift within the continental interior far from the continental margin and subsequent orogenesis. Paleomagnetic directional data from chapter 1 show that the anorthosite xenoliths as well as their host diabase rocks have unusually simple paleomagnetic behavior for their greater than 1 Ga age. 

They are attractive targets for paleomagnetic study as plagioclase crystals can protect magnetic inclusions from alteration. In addition, the alteration of the plagioclase crystals does not readily result in the formation of secondary iron oxides in contrast with Fe-silicate minerals such as olivine and pyroxene. There is a fruitful history of using silicate-hosted magnetic particles to reconstruct ancient paleointensity records. 

% replot Butler 1975 a single domain diagram
\begin{figure}
    \centering
    \includegraphics{}
    \caption{Caption}
    \label{fig:Butler_single_domain}
\end{figure}


% show example vortex-state grain and stability field 
\begin{figure}
    \centering
    \includegraphics{}
    \caption{Caption}
    \label{fig:micromagnetic_model}
\end{figure}


That we can recover paleomagnetic information thanks to the tiny grains of ferromagnetic minerals. In the classical rock magnetism domain theory, single domain 

Ferromagnetic minerals in rocks can record Earth's magnetic field directions. In igneous rocks, minerals such as Fe-Ti oxides and Fe-sulfides can preferentially align with local Earth magnetic field directions during cooling through blocking temperatures. In sedimentary rocks, it is the preferential alignment of magnetic grains during settling and deposition that lead to a detrital magnetic remanence in those rock records. By measuring the directions in rock records, we can reconstruct the location of past Earth's magnetic north pole position under the assumption that the Earth magnetic field is dominated by a dipole. 

To investigate the characterstics of Earth's geodynamo , I have developed both paleomagnetic directional and intensity data from the 1.1 billion-year-old anorthosite xenoliths of the Beaver Bay Complex in the Midcontinent Rift system that formed at the end of the Mesoproterozoic Era. I then develop paleomagnetic records from rocks in Death Valley and the Grand Canyon to provide another look at the Mesoproterozoic geomagneic field from a different location. 

In this dissertation I focus on exploring questions


While much of  the study and use of magnetic stones were for directional purposes, the modern scientific study of magnetism of ancient rocks, i.e. ``paleomagnetism" did not gain its momentum until the 20th century. Through the use of inclination data derived from oceanic crust, we know how plate tectonics operated and the relative motion of plates in the past ca. 200 Ma. With the use of full vector directional paleomagnetic data from older rocks on the continent, now we are able to recosntruction global paleogeography deeper back in time, and discovered that there have been times in Earth history where continents conjoined to form supercontinents. Furthermore, by reconstructing the intensity of the dipolar magnetic moment of the geomagnetic field through Earth history we are able to explore the evolution of Earth's thermal history as well as the change in its inner structure (e.g. the timing of existence of the inner core), which can help us better understand the formation of planets. 

While many topics are still in debate and better methods, and statistics are emerging to improve the statistics, we are in an era where enriching compilations of direction data and intensity data are marking exciting progress in understanding Earth history and understanding of planet formation. 

We are still in the debate of the birth timing of the inner core. 

related development in the study of magnetic material led to the invention of cassets tapes (a version of which google still use today for long-term data storage!). Recent years the study of electromagnetism has progressed from 3D bulk material to 2-dimensional materials. Recent discovery of altermagnetism as a new class of magnetic ordering keeps expanding our understanding of the fundamentals of the physical world and sheds new lights to material and theoretical development of the future. 



4. In the chapters of this thesis, I use paleomagnetic data developed from Mesoproterozoic rocks of the Midcontinent Rift to explore the paleogeography 


4. In \textbf{Chapter 2}, I develop paleomagnetic data from Mesoproterozoic-aged diabase and anorthosite xenoliths from the Midcontinent Rift that outcrop today along the stretch of the North Shore area to explore magmatic linkage between the intrusions and the Greenstone lava flow, a single lava flow up to 400 meters in thickness which outcrop along the Keweenaw Peninsula and Isle Royale. Only indirect evidence can be drawn since direct field relationship is covered by water and other rock units in the present-day basin of Lake Superior. I compare the paleomagnetic pole position recorded by the Beaver River diabase intrusions with that developed from the Greenstone flow to test the hypothesis that they are cogenetic from a temporal aspect. 

5. 


Measurements of paleomagnetic remanence are used to develop paleolatitude constraints using the working hypothesis that the surface expression of Earth’s geomagnetic field averages out to a geocentric axial dipole (GAD) where the full vector can be decomposed into two orthogonal components where one points toward the center of the Earth. 
Redrawn after McElhinny (1973).

Of particular use is that under this dipole model the inclination of magnetization (I) is a simple function of latitude (λ) that can be determined using the “dipole equation”( Fig. 1.3)):
tan(I) = 2 tan(λ)

modern day observations show that Earth's magnetic field expressed on the surface is constantly changing but conforms to a dipole pattern with the poles overlapping with geographic poles if the observations are averaged over a long enough period of time. Statistical analyses have shown that 



The Proterozoic Eon is a important middle stage of Earth's 4.5 billion years of life with a distinct role that connects the preceeding Archean Eon when continental lithosphere start to emerge, and the following Phanerozoic Eon when multicellular life emerged and flourished on the planet. ancient pieces of cratons almagamated into the Proterozoic Laurentia craton, substantial evidence of modern-day style plate tectonics, surface evolution of eukaryotes, etc. The evolution through these 2 billion years culminated in a climatic extrema called Snowball Earth followed by the Cambrian period where life exploded. The rocks formed through magmatism bear information of deep Earth properties, the resultant lithospheric plate tectonic configurations set the stage for surface evolution. This dissertation focus on developing data that constrain the characteristics of the magmatism, paleogeography configurations in an improved chronological context.


1. The path of the research trajectory is largely dynamically evolved through progress of developing robust paleomagnetic data and pair that with high-precision chronological context and use such data in tectonic and magmatic interpretations 

2. start the program with evaluating the magmatic correlation between the intrusive Beaver River diaabse and the extrusive Greenstone flow. 

3. 

the key to obtain robust paleomagnetic record and pair that with chronological context. The asymmetry of the Keweenawan Track is resolved through high-resolution paleomag in a stratrigraphic context. paleomag data and high-precision absolute geochronology data played crucial role in revealing the 

This dissertation has a focus on the paleomagnetism and specifically using paleomagnetic records in rocks formed in the ancient north American continent to explore magmatic linkage between Beaver river diabase and the Greenstone flow and the implication for the scale and intensity of mafic magmatism in the Midcontinent Rift ca. 1.1 billion years ago; using anorthosite xenoliths to reconstruct the intensity of Earth's magnetic field 1.1 billion years ago and the implication for the strength of the geodynamo at the time, and the implications for the status of Earth's core---in relation to the Earth's thermal history; 

Learning from the work pioneered by Ian Rose, a former Ph.D. student at Berkeley EPS advised by Professor Bruce Buffett and Nick Swanson-Hysell who developed a Bayesian method 

during the time I worked with an undergraduate James Pierce on an hornors thesis project where we investigated how hematite-rich sedimentary rocks record paleomagnetic field directions and developed a method to represent uncertainties associated with inclination shallowing in such sedimentary record. 


\subsection{Using detrital remanence magnetization in hematite-bearing sedimentary rocks for paleogeographic reconstructions}

Moving from a rich record of paleomagnetic data especially provided by the well-preserved Midcontinent Rift, we have a lull in well-dated paleomagnetic data in the $\sim$300 Myr afterwards, until the ca. 775 Ma Gunbarrel dikes (as compiled in \cite{Eyster2019a}). We have to use sedimentary rocks when there is a lack of igneous poles. 

The Jacobsville Formation is a good candidate as it is post-rift sedimentary formation in the rift that has been well preserved. With recent exposure as the Oronto Group has been dated to have beginning of exposure since the retreat of the Laurentide ice sheet. However, to be able to use sedimentary data in paleomagnetic pole compilation requires sedimentary rock-specific statistical tools as there are distinctions between uncertainties associated with sedimentary paleomagnetic directional data. In igneous rocks, directional data uncertainties are typically considered to be sourced from spherically randomly distributed error introduced during sample handling throughout the experimental procedures and can be propagated through the spherical Fisher statistics \cite{Fisher1953a}. In sedimentary rocks, rotation of magnetic grains during deposition and post-formational compaction process has been found to cause apparent shallowing in the recorded inclinations with respect to the inclinations of the geomagnetic fields they were deposited in. Such phenomenon is known as the ``inclination shallowing" and has been demonstrated experimentally through redeposition process of clastic sedimentary rocks. 

Through an undergraduate summer research project and through mentoring an undergraduate honors thesis project, I collaborated with James Pierce in 2022 and published a manuscript named ``Quantifying Inclination Shallowing and Representing Flattening Uncertainty in Sedimentary Paleomagnetic Poles" in \textit{Geochemistry, Geophysics, Geosystems} where proposed using the spherical bivariate Kent distribution to represent uncertainties associated with inclination shallowing in sedimentary rocks. 

This new tool has been used in the following reserach work coming out from our group, including \cite{Slotznick2023a} and \cite{Zhang2024a}. 

\subsection{The global paleogeography in the earliest Neoproterozoic: puzzles and progress}

With the newly developed statistical tool in hand, 

\end{abstract}

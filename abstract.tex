% (This file is included by thesis.tex; you do not latex it by itself.)

\begin{abstract}

The Proterozoic Eon is a period of Earth's history that began 2.5 billion years ago and ended 541 million years ago. During this time, stable continents first appeared and began to amalgamate, the first abundant fossils of living organisms started to appear in the geologic records, and the atmospheric concentration of oxygen started to rise from nearly null. At the boundary between the Mesoproterozoic (1.7 to 1.0 billion-years ago) and Neoproterozoic Eras (1.0 to 0.54 billion-years ago), the majority of the global continents conjoined together and formed a supercontinent called Rodinia, which eventually broke up throughout the Neoproterozoic Era. In this dissertation, I use field observations and paleomagnetic data in the ancient North American craton, Laurentia, to study its magmatic and tectonic history at the time and to investigate the geodynamic linkages between these lithospheric processes with deep-Earth evolution leading up to the assembly of Rodinia. 

I use paleomagnetic data in combination with geochronologic data and geochemical data to test a hypothesis that in the late Mesoproterozoic Midcontinent Rift, the voluminous Beaver River diabase that outcrop in Minnesota, US along Lake Superior was the feeder system for one of the largest lava flows known---the Greenstone Flow, which outcrops in Isle Royale and in Keweenaw Peninsula, Michigan. I show that the diabase and the lava flow were emplaced synchronously and share chemical and petrological characteristics. Together, the extrusive and intrusive rocks define an episode of rapid and voluminous mafic magmatism ca. 1092 Ma as Laurentia rapidly traveled from high latitudes toward the equator. 

The large and well-preserved anorthosite xenoliths within the Beaver River diabase not only serve as evidence for wide magma conduits and vigorous magmatism at the time, but also provide an opportunity to directly reconstruct the strengths of Earth's surface magnetic field in the Mesoproterozoic, thanks to their unusual mineralogical assemblage. Previously, it was proposed that Earth did not have a solid inner core until the Ediacaran Period, and the Proterozoic Eon witnessed a progressive decay of Earth's magnetic field strength due to the decrease in the activity of fluid convection in the core. I developed paleointensity data from the Beaver River anorthosite xenoliths. Instead of showing weak geomagnetic field strengths as predicted by the hypothesis, the data reveal a strong geodynamo at the time. I argue that the high geomagnetic field intensities shown by the anorthosites indicate that there were strong power sources to the geodynamo in the late Mesoproterozoic. Whether or not such power source include compositional convection driven by a substantial solid inner core requires more paleomagnetic data, geodynamo modeling, and advances in material property characterization at Earth's core conditions. 

While protracted magmatism occurred in the Midcontinent Rift region in the late Mesoproterozoic, mafic magmatism also occurred in southwestern Laurentia. I report new paleomagnetic data paired with high-precision zircon U-Pb geochronology data and investigate the temporal magmatic correlations between the two regions. I show that the rapid emplacement of thick mafic intrusions throughout a large extent in southwestern Laurentia define a large igneous province ca. 1098 Ma. The timing and scale of this episode of magmatism allows for a geodynamic linkage between it and the large igneous province associated with the Duluth Complex in the Midcontinent Rift, where a mantle plume first arrived and resulted in widespread magmatism in the southwest, before draining toward the Midcontinent Rift where it drove the emplacement of the Duluth Complex and associated volcanism. 

Lastly, I report new paleomagnetic data from sedimentary rocks of the Jacobsville Formation for constraining the global paleogeography in the earliest Neoproterozoic. Clastic sedimentary rocks are an important archive of Earth history. Detrital hematite-bearing red bed sedimentary rocks are particularly useful for providing crucial paleogeographic constraints. With the red siltstone and fine-grained sandstone facies of the Jacobsville Formation I show that Laurentia's plate speed slowed down by an order of magnitude in the late Mesoproterozoic to early Neoproterozoic following the onset of the Grenvillian orogeny. I further hypothesize that the previous estimates of the ages of the paleomagnetic poles of the Grenville Loop need to be calibrated toward much younger.

\end{abstract}

% (This file is included by thesis.tex; you do not latex it by itself.)

\begin{abstract}

% \begin{center}
%     \textit{``An ancient time\\
%     When compass needles lost their way; \\
%     Finding North from South." \\}
%     \vspace{10pt}
%     S. W. Bogue
% \end{center}

Over 2000 years ago, one of humanity's earliest records of harnessing magnetism unfolded in ancient China, where lodestones (i.e. magnetic rocks) were used in telling directions in geomancy and divination practices. Compass made its way to Europe through the Silk Road, which granted European mariners the ability to tell their heading regardless of weather conditions or visibility constraints at sea. The improved navigation for mariners contributed to the Age of Discovery. Scientific advances through the centuries have made profound insights into the fundamental principles of electromagnetism. Today, our continued understanding and manipulation of the electromagnetic principles lie at the core of modern technology and life. 

It is because compass needles are made of magnetic materials that tend to align with Earth's magnetic field at surface which is dominantly dipolar and stable such that a compass can help navigate today. In the rock records, there are numerous microscopic compasses that are carrying ancient magnetic records through deep time. These microscopic compasses are magnetic minerals such as iron-titanium oxide minerals and iron sulfide minerals that occur rather ubiquitously in igneous, sedimentary, and metamorphic rocks. Through much of Earth's history, that a geodynamo that powered a strong magnetic field The study of paleomagnetism explores Earth history stories while being fundamentally rooted in the understanding of rock forming processes and the physics of magnetism.

In this dissertation, I find my applications of paleomagnetism in exploring the characteristics of magmatism in the Midcontinent Rift system in the Lake Superior region and in the southwestern Laurentia 1.1 billion years ago, shedding new lights in the strength of the geodynamo ca. 1092 million years ago, and furthering the development of global paleogeography through the end of the Mesoproterozoic and the earliest Neoproterozoic. 

\subsection{Late Mesoproterozoic magmatism in the Midcontinent Rift and the southwestern Laurentia large igneous province}

The ancient North American craton, Laurentia, first formed in the Paleoproterozoic Era, when a series of collisional orogenies culminating the Trans-Hudson orogeny led to the amalgamation of Archean provinces \cite{Hoffman1988a, Whitmeyer2007a}. The craton continued to grow through the rest of the Paleoproterozoic and into the Mesoproterozoic via accretionary orogenesis along its margin. In the latest Mesoproterozoic (ca. 1110 to 1080 Ma) the large intracontinental Midcontinent Rift, which was colocated with a large igneous province (LIP) \cite{Swanson-Hysell2021a}, formed in Laurentia’s interior, leading to extension within the Archean Superior province and adjacent Paleoproterozoic provinces to the south \cite{Cannon1992a}. Protracted magmatic activities punctuated by rapid and voluminous emplacement of extrusive and intrusive rocks in the rift led to the emplacement of a thick succession of volcanic rocks and mafic intrusions in Laurentia’s interior. Thermal subsidence syn- to post-rift led to the deposition of clastic sedimentary rocks on top of the igneous rocks.

However, the Midcontinent Rift eventually ceased and failed to split the continent into two. Far-field compressional forces associated with the onset and development of the Grenvillian orogeny along the eastern margin of Laurentia led to cessation and subsequent inversion of the rift \cite{Cannon1994a, Swanson-Hysell2019a}. In the southern Lake Superior region, Midcontinent Rift volcanic and sedimentary rocks were uplifted along with Paleoproterozoic and Archean lithologies via thrust faults, forming the crustal-scale Montreal River monocline. The resultant topography led to the deposition of the early Neoproterozoic Jacobsville Formation, which overlies an angular unconformity that developed on lithologies that were exhumed through this earlier episode of contractional deformation associated with Grenvillian orogenesis. 

The intracontinental nature of the rift and the its cessation led to large amount of rift rocks being preserved in the interior of Laurentia far from continental margins. Rocks in the rift were subsequently experienced mild burial and metamorphism history. Rb-Sr dates from the uplifted basement rocks in southern Lake Superior region show that the area has not been heated to above 300 \textdegree C since ca. 1050 Ma \cite{Cannon1993a}. $^{10}$Be data show that some surface bedrock exposures have only recently been near the surface due to Pleistocene glacial and recent fluvial erosion \cite<e.g.>{Ullman2015a}. 

The rich record of well-preserved Midcontinent Rift rocks provide a wealth of opportunities for characterizing the magmatic history of Laurentia at the time. Geochronology and geochemistry data have been used to divide magmatic activity in the rift into four stages based on interpreted changes in relative magmatic volume and the nature of magmatism: early ($\sim$1109–1104 Ma), latent ($\sim$1104–1098 Ma), main ($\sim$1098–1090 Ma) and late ($\sim$1090–1083 Ma) \cite{Vervoort2007a, Heaman2007a, Miller2013a}. Advances in the method of chemical abrasion-isotope dilution-thermal ionization mass spectroscopy (CA-ID-TIMS) and its application in obtaining high-precision zircon U-Pb ages, together with the development of high-quality paleomagnetic data from the extrusive and intrusive rocks, have revealed distinct rapid and voluminous episodes of magmatism in the rift during the overall protracted period of active magmatism \cite{Swanson-Hysell2019a, Swanson-Hysell2021a, Zhang2021b}. In particular, a large igneous province, consisting of the massive Duluth Complex and its associated $\sim$8 km thick extrusive lava flows of the North Shore Volcanic Group, formed within 500 Kyr ca. 1096 Ma \cite{Swanson-Hysell2021a}. Stratigraphically above the Duluth Complex, the ca. 1092 Ma mafic Beaver Bay Complex represents another rapid and voluminous pulse of magmatism. It features the emplacement of the hypabyssal intrusion of the Beaver River diabase that has wide magma conduits containing giant anorthosite xenoliths that can be as much as 300 meters in width. Unlike the Duluth Complex and the North Shore Volcanic Group, whose magmatic linkage is supported by the exposed stratigraphic relationships and geochronologic data, there is no direct exposure of volcanic rocks that overlie conformably on top of the Beaver River diabase. Previously on the basis of geochemical data and petrographic observations, a hypothesis was put forward that the Greenstone Flow of the Portage Lake Volcanics that outcrop in the upper Keweenaw Peninsula in Michigan could be the surface expression of the Beaver River diabase. In chapter 1, I test this hypothesis by using paleomagnetic and geochronologic data to investigate the synchroneity of the emplacement of the intrusive and extrusive rocks. The data support that the Beaver River diabase was the feeder system for the $\sim$400 meters thick Greenstone Flow which ranks one of the largest lava flows known on Earth. 

While rapid and voluminous mafic magmatism occurred in the Midcontinent Rift, improved geochronologic and paleomagnetic data show that distinct episodes of mafic magmatism occurred in southwestern Laurentia ca. 1098 Ma and ca. 1082 Ma \cite{Mohr2024a}. During a period of $<$0.25 Myr ca. 1098 Ma, thick mafic diabase sills and dikes intruded through southwestern basement rocks into the Mesoproterozoic sedimentary rocks including the Crystal Spring Formation in the Death Valley, Unkar Group sedimentary rocks in the Grand Canyon, and the Apache Group sedimentary rocks in central Arizona \cite{Bright2014a}. The tempo and extent of this episode of mafic magmatism and juvenile geochemical signature of the rocks are consistent with there being a southwestern large igneous province driven by a mantle plume. That the emplacement of the voluminous intrusions in the southwest occurred 2 Myr prior to the Duluth Complex led \cite{Mohr2024a} to invoke a tectonic model, where the mantle plume first arrived ca. 1098 Ma in southwestern Laurentia and laterally transported toward the readily thinned crust of the Midcontinent Rift. The arrival of the plume material in the rift postdates the latent magmatic stage ($\sim$1104–1098 Ma) and led to a replenishment of mafic magma and heat supply in the rift, which eventually drove the emplacement of the ca. 1096 Ma Duluth Complex and the associated lava flows. In chapter 3 I develop new paleomagnetic data from southwestern Laurentia and update the extent of the southwestern large igneous province. 

A fruitful history of the development of paleomagnetic and high-precision geochronology data has helped advance the understanding of late Mesoproterozoic plate tectonics and Earth interior processes. Syntheses of paleomagnetic directional data in detailed volcanostratigraphic context, and pairing such data with high-precision geochronology data has found that Earth's magnetic field at the time was dominantly dipolar with symmetric reversals \cite{Swanson-Hysell2014a}, had a paleosecular variation patter similar to that today \cite{Tauxe2009a}, and likely developed superchrons \cite{Driscoll2016b}. The rapid pole progression from ca. 1110 Ma to ca. 1080 Ma shown by the Keweenawan Track indicate a period of rapid plate tectonic motion of Laurentia leading up to the Grenville continental collisional orogenesis and the assembly of the supercontinent Rodinia \cite{Swanson-Hysell2019a, Swanson-Hysell2023a, Rose2022a}. In this dissertation, I develop new paleomagnetic directional data from the Midcontinent Rift and the southwestern Laurentia to further constrain the apparent polar wander path of Laurentia in the late Mesoproterozoic. I utilize the post-rift Jacobsville Formation in the Midcontinent Rift to fill the gap of paleomagnetic data in the earliest Neoproterozoic and constrain the paleogeography of Rodinia at the time. I also develop new paleomagnetic intensity data from anorthosite xenoliths in the Midcontinent Rift to probe the strength of the geodynamo ca. 1092 Ma. 

\subsection{The geocentric axial-dipole hypothesis in the late Mesoproterozoic}

It is the dipole hypothesis that grants power to paleomagnetic directional data for resolving temporal linkage between rocks. 

Ferromagnetic minerals in rocks can record Earth's magnetic field directions. In igneous rocks, minerals such as Fe-Ti oxides and Fe-sulfides can preferentially align with local Earth magnetic field directions during cooling through blocking temperatures. In sedimentary rocks, it is the preferential alignment of magnetic grains during settling and deposition that lead to a detrital magnetic remanence in those rock records. By measuring the directions in rock records, we can reconstruct the location of past Earth's magnetic north pole position under the assumption that the Earth magnetic field is dominated by a dipole. 

One of the primary methods of using paleomagnetic directional data to resolve temporal relationship between rock units is to compare paleomagnetic pole positions. Overlapping pole positions would imply similar age while distinct pole positions would indicate different ages. Such paleomagnetic test requires that Earth's magnetic field was dipolar and that paleosecular variation is averaged out. 

introduction to the GAD hypothesis
    Earth's magnetic field is the result of convective flow of liquid iron-alloy in Earth's outer core. At present day, the geodynamo is collectively powered by heat flow across the core-mantle boundary (CMB) and from the crystallization of the solid inner core from the liquid outer core which provides latent heat and compositional buoyancy due to the exclusion of light elements \cite{Buffett2000a}. Studies have found that a dynamo field has existed since at least 3.4 billion years ago \cite{Selkin2007a, Biggin2011a, Tarduno2014a, Brenner2020a}.
    
    dipole equation
    importance of averaging out paleosecular variation
    verified in the past 5 Myr
    the TK03 model developed using volcanic data in the past 5 myr
    north shore data has been shown to conform to such a model, suggesting a GAD dominated field similar to that of recent Cenozoic time scale ca. 1.1 Ga
    additionally, Pierce et al., we showed that detrital hematite bearing paleosol sedimentary rocks of the Midcontinent Rift bracketed by north shore volcanic lava flows using the TK03 model to correct for detrital remanence directions agrees very well to the lava flow directions. This also support a GAD field model during the Midcontinent Rift time.
    Additional paleomagnetic field tests that support a GAD dominated field include the reversal test in detailed stratigraphic context of Swanson-Hysell publications. 

while there are many paleomagnetic directional data in the late Mesoproterozoic, few high-quality paleomagnetic intensity data exist at the time. It is essential to obtain intensity data because they  help measure the strength of the geodynamo and are one of the only direct observational data that shed light into the thermal history of the Earth. With the timing of inner core nucleation still at debate due to large uncertainty in the physical properties of core materials including the thermal conductivity, theoretical calculations have difficulty differentiating whether Earth had a solid inner core since the primordial time or it formed later in the history. 

\subsection{Probing Earth's core via surface magnetic field observations}

can also cite Daria's work here

\subsection{Recovering deep time full-vector paleomagnetic records using silicate-hosted Fe-oxides}

That we can recover paleomagnetic information thanks to the tiny grains of ferromagnetic minerals. In the classical rock magnetism domain theory, single domain 

There is a wealth of paleomagnetic directional data from the Midcontinent Rift system, a vast geologic province covering much of the area around Lake Superior today and extends in two branches toward Kansas and Michigan. Protracted magmatism and subsequent failure in rifting ancient North America continent (Laurentia) apart resulted in a rich rock record in the rift that is well-preserved through the following billion years. \cite{Swanson-Hysell2019a} developed and compiled state-of-the-art paleomagnetic directional and high-precision geochronology data from the Midcontinent Rift which has become the central record of global paleogeography reconstruction in the late Mesoproterozoic Era \cite{Evans2021a}. However, 

To investigate the characterstics of Earth's geodynamo , I have developed both paleomagnetic directional and intensity data from the 1.1 billion-year-old anorthosite xenoliths of the Beaver Bay Complex in the Midcontinent Rift system that formed at the end of the Mesoproterozoic Era. I then develop paleomagnetic records from rocks in Death Valley and the Grand Canyon to provide another look at the Mesoproterozoic geomagneic field from a different location. 

In this dissertation I focus on exploring questions


While much of  the study and use of magnetic stones were for directional purposes, the modern scientific study of magnetism of ancient rocks, i.e. ``paleomagnetism" did not gain its momentum until the 20th century. Through the use of inclination data derived from oceanic crust, we know how plate tectonics operated and the relative motion of plates in the past ca. 200 Ma. With the use of full vector directional paleomagnetic data from older rocks on the continent, now we are able to recosntruction global paleogeography deeper back in time, and discovered that there have been times in Earth history where continents conjoined to form supercontinents. Furthermore, by reconstructing the intensity of the dipolar magnetic moment of the geomagnetic field through Earth history we are able to explore the evolution of Earth's thermal history as well as the change in its inner structure (e.g. the timing of existence of the inner core), which can help us better understand the formation of planets. 

While many topics are still in debate and better methods, and statistics are emerging to improve the statistics, we are in an era where enriching compilations of direction data and intensity data are marking exciting progress in understanding Earth history and understanding of planet formation. 

We are still in the debate of the birth timing of the inner core. 

related development in the study of magnetic material led to the invention of cassets tapes (a version of which google still use today for long-term data storage!). Recent years the study of electromagnetism has progressed from 3D bulk material to 2-dimensional materials. Recent discovery of altermagnetism as a new class of magnetic ordering keeps expanding our understanding of the fundamentals of the physical world and sheds new lights to material and theoretical development of the future. 

I find much joy in studying in depth the field of paleomagnetism while being exposed to related topics in material science through collaborations and daily interactions with colleagues. In the following part of the I will have a brief introduction about our understanding of Earth's geodynamo, the ways in which rocks may acquire paleomagnetic records, and they way I use those records to explore questions in Earth history in the following individual project-based chapters. 




4. With the GAD model we can reconstruct paleogeography in deep time before the earliest available seafloor magnetic data. 

5. The GAD model not only gives a quantitative relationship between paleolatitude and inclination, the dipole field model also given quantitative projection of surface field intensity and the geodynamo dipole moment. The measured field also has latitudinal dependency. By converting field intensity observations at a certain paleolatitude into paleodynamo dipole moments, this construct allows us to establish a record of the strength of geodynamo through Earth history and which can allow us to explore the evolution of the structure and thermal history of the interior of Earth through time. 


4. In the chapters of this thesis, I use paleomagnetic data developed from Mesoproterozoic rocks of the Midcontinent Rift to explore the paleogeography 


4. In \textbf{Chapter 2}, I develop paleomagnetic data from Mesoproterozoic-aged diabase and anorthosite xenoliths from the Midcontinent Rift that outcrop today along the stretch of the North Shore area to explore magmatic linkage between the intrusions and the Greenstone lava flow, a single lava flow up to 400 meters in thickness which outcrop along the Keweenaw Peninsula and Isle Royale. Only indirect evidence can be drawn since direct field relationship is covered by water and other rock units in the present-day basin of Lake Superior. I compare the paleomagnetic pole position recorded by the Beaver River diabase intrusions with that developed from the Greenstone flow to test the hypothesis that they are cogenetic from a temporal aspect. 

5. 


Measurements of paleomagnetic remanence are used to develop paleolatitude constraints using the working hypothesis that the surface expression of Earth’s geomagnetic field averages out to a geocentric axial dipole (GAD) where the full vector can be decomposed into two orthogonal components where one points toward the center of the Earth. 
Redrawn after McElhinny (1973).

Of particular use is that under this dipole model the inclination of magnetization (I) is a simple function of latitude (λ) that can be determined using the “dipole equation”( Fig. 1.3)):
tan(I) = 2 tan(λ)

modern day observations show that Earth's magnetic field expressed on the surface is constantly changing but conforms to a dipole pattern with the poles overlapping with geographic poles if the observations are averaged over a long enough period of time. Statistical analyses have shown that 



The Proterozoic Eon is a important middle stage of Earth's 4.5 billion years of life with a distinct role that connects the preceeding Archean Eon when continental lithosphere start to emerge, and the following Phanerozoic Eon when multicellular life emerged and flourished on the planet. ancient pieces of cratons almagamated into the Proterozoic Laurentia craton, substantial evidence of modern-day style plate tectonics, surface evolution of eukaryotes, etc. The evolution through these 2 billion years culminated in a climatic extrema called Snowball Earth followed by the Cambrian period where life exploded. The rocks formed through magmatism bear information of deep Earth properties, the resultant lithospheric plate tectonic configurations set the stage for surface evolution. This dissertation focus on developing data that constrain the characteristics of the magmatism, paleogeography configurations in an improved chronological context.


1. The path of the research trajectory is largely dynamically evolved through progress of developing robust paleomagnetic data and pair that with high-precision chronological context and use such data in tectonic and magmatic interpretations 

2. start the program with evaluating the magmatic correlation between the intrusive Beaver River diaabse and the extrusive Greenstone flow. 

3. 

the key to obtain robust paleomagnetic record and pair that with chronological context. The asymmetry of the Keweenawan Track is resolved through high-resolution paleomag in a stratrigraphic context. paleomag data and high-precision absolute geochronology data played crucial role in revealing the 

This dissertation has a focus on the paleomagnetism and specifically using paleomagnetic records in rocks formed in the ancient north American continent to explore magmatic linkage between Beaver river diabase and the Greenstone flow and the implication for the scale and intensity of mafic magmatism in the Midcontinent Rift ca. 1.1 billion years ago; using anorthosite xenoliths to reconstruct the intensity of Earth's magnetic field 1.1 billion years ago and the implication for the strength of the geodynamo at the time, and the implications for the status of Earth's core---in relation to the Earth's thermal history; 

Learning from the work pioneered by Ian Rose, a former Ph.D. student at Berkeley EPS advised by Professor Bruce Buffett and Nick Swanson-Hysell who developed a Bayesian method 

during the time I worked with an undergraduate James Pierce on an hornors thesis project where we investigated how hematite-rich sedimentary rocks record paleomagnetic field directions and developed a method to represent uncertainties associated with inclination shallowing in such sedimentary record. 


\subsection{The merits and caveats in using hematite-carrying detrital remanence magnetization for paleogeographic reconstructions}

Moving from a rich record of paleomagnetic data especially provided by the well-preserved Midcontinent Rift, we have a lull in well-dated paleomagnetic data in the $\sim$300 Myr afterwards, until the ca. 775 Ma Gunbarrel dikes (as compiled in \cite{Eyster2019a}). We have to use sedimentary rocks when there is a lack of igneous poles. 

The Jacosbville Formation is a good candidate as it is post-rift sedimentary formation in the rift that has been well preserved. With recent exposure as the Oronto Group has been dated to have beginning of exposure since the retreat of the Laurentide ice sheet. However, to be able to use sedimentary data in paleomagnetic pole compilation requires sedimentary rock-specific statistical tools as there are distinctions between uncertainties associated with sedimentary paleomagnetic directional data. In igneous rocks, directional data uncertainties are typically considered to be sourced from spherically randomly distributed error introduced during sample handling throughout the experimental procedures and can be propagated through the spherical Fisher statistics \cite{Fisher1953a}. In sedimentary rocks, rotation of magnetic grains during deposition and post-formational compaction process has been found to cause apparent shallowing in the recorded inclinations with respect to the inclinations of the geomagnetic fields they were deposited in. Such phenomenon is known as the ``inclination shallowing" and has been demonstrated experimentally through redeposition process of clastic sedimentary rocks. 

Through an undergraduate summer research project and through mentoring an undergraduate honors thesis project, I collaborated with James Pierce in 2022 and published a manuscript named ``Quantifying Inclination Shallowing and Representing Flattening Uncertainty in Sedimentary Paleomagnetic Poles" in \textit{Geochemistry, Geophysics, Geosystems} where proposed using the spherical bivariate Kent distribution to represent uncertainties associated with inclination shallowing in sedimentary rocks. 

This new tool has been used in the following reserach work coming out from our group, including \cite{Slotznick2023a} and \cite{Zhang2024a}. 

\subsection{The global paleogeography in the earliest Neoproterozoic: puzzles and progress}

With the newly developed statistical tool in hand, 

\end{abstract}
